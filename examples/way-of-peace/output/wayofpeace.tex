\documentclass[12pt,oneside]{scrbook}


% Fonts and typography

%% Typography
\usepackage[no-math]{fontspec}
\defaultfontfeatures{Scale = 1}

%% Fonts
\setmainfont[Ligatures=TeX]{Verdana}
\setsansfont[Ligatures=TeX]{Cambria Bold Italic}
\setmonofont{Consolas}

% Section fonts
\usepackage{sectsty}
\allsectionsfont{\sffamily}
\subsectionfont{\sffamily}
\chapterfont{\LARGE\sffamily}


%% Set polyglossia language
\usepackage{polyglossia}
\setdefaultlanguage{english}


% Page

%% Use full page in book style
\usepackage{fullpage}

%% Set line spacing
\usepackage{setspace}
\setstretch{1.2}

%% Disable paragraph indentation
\usepackage{parskip}

% Colors

\usepackage{xcolor}

%% Tango color scheme
\definecolor{SkyBlue}{HTML}{3465A4}
\definecolor{DarkSkyBlue}{HTML}{204A87}

\definecolor{Plum}{HTML}{75507B}

\definecolor{ScarletRed}{HTML}{CC0000}


\definecolor{Aluminium1}{HTML}{EEEEEC}
\definecolor{Aluminium6}{HTML}{2e3436}

\definecolor{Black}{HTML}{000000}

% Use fancy chapter numbering and set the color
\usepackage{quotchap}
\definecolor{chaptergrey}{HTML}{204A87}

% Listings
\usepackage{upquote}

\usepackage{listings}

\lstdefinelanguage{JavaScript}{
  keywords = {typeof, new, true, false, catch, function, return, null, catch, switch, var, if, in, while, do, else, case, break},
  keywordstyle = \color{SkyBlue}\bfseries,
  ndkeywords = {class, export, boolean, throw, implements, import, this},
  ndkeywordstyle = \color{Aluminium6}\bfseries,
  identifierstyle = \color{Black},
  sensitive = false,
  comment = [l]{//},
  morecomment = [s]{/*}{*/},
  commentstyle = \color{Plum}\ttfamily,
  stringstyle = \color{ScarletRed}\ttfamily,
  morestring = [b]',
  morestring = [b]"
}

\lstset{
  language = JavaScript,
  backgroundcolor = \color{Aluminium1},
  extendedchars = true,
  basicstyle = \normalsize\ttfamily,
  showstringspaces = false,
  showspaces = false,
  tabsize = 1,
  breaklines = true,
  showtabs = false
}


% Links

%% Hyperref
\usepackage[colorlinks, breaklinks, bookmarks, xetex]{hyperref}

\hypersetup {
  pdfauthor={James Allen},
  pdftitle={The Way of Peace},
  linkcolor = DarkSkyBlue,
  citecolor = DarkSkyBlue,
  filecolor = DarkSkyBlue,
  urlcolor = DarkSkyBlue
}

%% Don’t use Mono font for URLs
\urlstyle{same}


% Images

\usepackage{graphicx}
\usepackage{incgraph}

\usepackage{eso-pic}
\newcommand\Cover{%
    \put(0,0){%
        \parbox[b][\paperheight]{\paperwidth}{%
        \vfill
        \centering
        \includegraphics[height=\paperheight,width=\paperwidth]{cover.jpg}
        \vfill
}}}


% Pandoc hacks

%% Normal enumerates processing
\usepackage{enumerate}

%% Disable section numbers
\setcounter{secnumdepth}{0}

% Makes table of contents look nicer
\usepackage{tocstyle}
\usetocstyle{standard}

% Set level of sections to be shown in Table of Contents
% 0 = chapters, 1 = chapters/sections, 2 = chapters/sections/subsections
\setcounter{tocdepth}{1}

\begin{document}

  % Title page
  \AddToShipoutPicture*{\Cover} % * = only use it on one page
  \ClearShipoutPicture
  
  \null\newpage
  
  \mainmatter
  \phantomsection  
  \addcontentsline{toc}{chapter}{Contents}
  \renewcommand{\contentsname}{Contents}
  \tableofcontents
  
  % Book contents

  \section*{Preface}\label{preface}
  \addcontentsline{toc}{section}{Preface}
  
  The Project Gutenberg EBook of The Way of Peace, by James Allen
  
  This eBook is for the use of anyone anywhere at no cost and with almost
  no restrictions whatsoever. You may copy it, give it away or re-use it
  under the terms of the Project Gutenberg License included with this
  eBook or online at www.gutenberg.net
  
  Title: The Way of Peace
  
  Author: James Allen
  
  Release Date: January 18, 2004 {[}EBook \#10740{]}
  
  Language: English
  
  *** START OF THIS PROJECT GUTENBERG EBOOK THE WAY OF PEACE ***
  
  Produced by Kevin Handy and PG Distributed Proofreaders
  
  \section*{Introduction}\label{introduction}
  \addcontentsline{toc}{section}{Introduction}
  
  THE WAY OF PEACE
  
  BY JAMES ALLEN
  
  AUTHOR OF ``AS A MAN THINKETH,'' ``OUT FROM THE HEART''
  
  \section{THE POWER OF MEDITATION}\label{the-power-of-meditation}
  
  Spiritual meditation is the pathway to Divinity. It is the mystic ladder
  which reaches from earth to heaven, from error to Truth, from pain to
  peace. Every saint has climbed it; every sinner must sooner or later
  come to it, and every weary pilgrim that turns his back upon self and
  the world, and sets his face resolutely toward the Father's Home, must
  plant his feet upon its golden rounds. Without its aid you cannot grow
  into the divine state, the divine likeness, the divine peace, and the
  fadeless glories and unpolluting joys of Truth will remain hidden from
  you.
  
  Meditation is the intense dwelling, in thought, upon an idea or theme,
  with the object of thoroughly comprehending it, and whatsoever you
  constantly meditate upon you will not only come to understand, but will
  grow more and more into its likeness, for it will become incorporated
  into your very being, will become, in fact, your very self. If,
  therefore, you constantly dwell upon that which is selfish and debasing,
  you will ultimately become selfish and debased; if you ceaselessly think
  upon that which is pure and unselfish you will surely become pure and
  unselfish.
  
  Tell me what that is upon which you most frequently and intensely think,
  that to which, in your silent hours, your soul most naturally turns, and
  I will tell you to what place of pain or peace you are traveling, and
  whether you are growing into the likeness of the divine or the bestial.
  
  There is an unavoidable tendency to become literally the embodiment of
  that quality upon which one most constantly thinks. Let, therefore, the
  object of your meditation be above and not below, so that every time you
  revert to it in thought you will be lifted up; let it be pure and
  unmixed with any selfish element; so shall your heart become purified
  and drawn nearer to Truth, and not defiled and dragged more hopelessly
  into error.
  
  Meditation, in the spiritual sense in which I am now using it, is the
  secret of all growth in spiritual life and knowledge. Every prophet,
  sage, and savior became such by the power of meditation. Buddha
  meditated upon the Truth until he could say, ``I am the Truth.'' Jesus
  brooded upon the Divine immanence until at last he could declare, ``I
  and my Father are One.''
  
  Meditation centered upon divine realities is the very essence and soul
  of prayer. It is the silent reaching of the soul toward the Eternal.
  Mere petitionary prayer without meditation is a body without a soul, and
  is powerless to lift the mind and heart above sin and affliction. If you
  are daily praying for wisdom, for peace, for loftier purity and a fuller
  realization of Truth, and that for which you pray is still far from you,
  it means that you are praying for one thing while living out in thought
  and act another. If you will cease from such waywardness, taking your
  mind off those things the selfish clinging to which debars you from the
  possession of the stainless realities for which you pray: if you will no
  longer ask God to grant you that which you do not deserve, or to bestow
  upon you that love and compassion which you refuse to bestow upon
  others, but will commence to think and act in the spirit of Truth, you
  will day by day be growing into those realities, so that ultimately you
  will become one with them.
  
  He who would secure any worldly advantage must be willing to work
  vigorously for it, and he would be foolish indeed who, waiting with
  folded hands, expected it to come to him for the mere asking. Do not
  then vainly imagine that you can obtain the heavenly possessions without
  making an effort. Only when you commence to work earnestly in the
  Kingdom of Truth will you be allowed to partake of the Bread of Life,
  and when you have, by patient and uncomplaining effort, earned the
  spiritual wages for which you ask, they will not be withheld from you.
  
  If you really seek Truth, and not merely your own gratification; if you
  love it above all worldly pleasures and gains; more, even, than
  happiness itself, you will be willing to make the effort necessary for
  its achievement.
  
  If you would be freed from sin and sorrow; if you would taste of that
  spotless purity for which you sigh and pray; if you would realize wisdom
  and knowledge, and would enter into the possession of profound and
  abiding peace, come now and enter the path of meditation, and let the
  supreme object of your meditation be Truth.
  
  At the outset, meditation must be distinguished from \emph{idle
  reverie}. There is nothing dreamy and unpractical about it. It is
  \emph{a process of searching and uncompromising thought which allows
  nothing to remain but the simple and naked truth}. Thus meditating you
  will no longer strive to build yourself up in your prejudices, but,
  forgetting self, you will remember only that you are seeking the Truth.
  And so you will remove, one by one, the errors which you have built
  around yourself in the past, and will patiently wait for the revelation
  of Truth which will come when your errors have been sufficiently
  removed. In the silent humility of your heart you will realize that
  
  \begin{quote}
  ``There is an inmost centre in us all Where Truth abides in fulness; and
  around, Wall upon wall, the gross flesh hems it in; This perfect, clear
  perception, which is Truth, A baffling and perverting carnal mesh Blinds
  it, and makes all error; and to know, Rather consists in opening out a
  way Whence the imprisoned splendour may escape, Than in effecting entry
  for a light Supposed to be without.''
  \end{quote}
  
  Select some portion of the day in which to meditate, and keep that
  period sacred to your purpose. The best time is the very early morning
  when the spirit of repose is upon everything. All natural conditions
  will then be in your favor; the passions, after the long bodily fast of
  the night, will be subdued, the excitements and worries of the previous
  day will have died away, and the mind, strong and yet restful, will be
  receptive to spiritual instruction. Indeed, one of the first efforts you
  will be called upon to make will be to shake off lethargy and
  indulgence, and if you refuse you will be unable to advance, for the
  demands of the spirit are imperative.
  
  To be spiritually awakened is also to be mentally and physically
  awakened. The sluggard and the self-indulgent can have no knowledge of
  Truth. He who, possessed of health and strength, wastes the calm,
  precious hours of the silent morning in drowsy indulgence is totally
  unfit to climb the heavenly heights.
  
  He whose awakening consciousness has become alive to its lofty
  possibilities, who is beginning to shake off the darkness of ignorance
  in which the world is enveloped, rises before the stars have ceased
  their vigil, and, grappling with the darkness within his soul, strives,
  by holy aspiration, to perceive the light of Truth while the unawakened
  world dreams on.
  
  \begin{quote}
  ``The heights by great men reached and kept, Were not attained by sudden
  flight, But they, while their companions slept, Were toiling upward in
  the night.''
  \end{quote}
  
  No saint, no holy man, no teacher of Truth ever lived who did not rise
  early in the morning. Jesus habitually rose early, and climbed the
  solitary mountains to engage in holy communion. Buddha always rose an
  hour before sunrise and engaged in meditation, and all his disciples
  were enjoined to do the same.
  
  If you have to commence your daily duties at a very early hour, and are
  thus debarred from giving the early morning to systematic meditation,
  try to give an hour at night, and should this, by the length and
  laboriousness of your daily task be denied you, you need not despair,
  for you may turn your thoughts upward in holy meditation in the
  intervals of your work, or in those few idle minutes which you now waste
  in aimlessness; and should your work be of that kind which becomes by
  practice automatic, you may meditate while engaged upon it. That eminent
  Christian saint and philosopher, Jacob Boehme, realized his vast
  knowledge of divine things whilst working long hours as a shoemaker. In
  every life there is time to think, and the busiest, the most laborious
  is not shut out from aspiration and meditation.
  
  Spiritual meditation and self-discipline are inseparable; you will,
  therefore, commence to meditate upon yourself so as to try and
  understand yourself, for, remember, the great object you will have in
  view will be the complete removal of all your errors in order that you
  may realize Truth. You will begin to question your motives, thoughts,
  and acts, comparing them with your ideal, and endeavoring to look upon
  them with a calm and impartial eye. In this manner you will be
  continually gaining more of that mental and spiritual equilibrium
  without which men are but helpless straws upon the ocean of life. If you
  are given to hatred or anger you will meditate upon gentleness and
  forgiveness, so as to become acutely alive to a sense of your harsh and
  foolish conduct. You will then begin to dwell in thoughts of love, of
  gentleness, of abounding forgiveness; and as you overcome the lower by
  the higher, there will gradually, silently steal into your heart a
  knowledge of the divine Law of Love with an understanding of its bearing
  upon all the intricacies of life and conduct. And in applying this
  knowledge to your every thought, word, and act, you will grow more and
  more gentle, more and more loving, more and more divine. And thus with
  every error, every selfish desire, every human weakness; by the power of
  meditation is it overcome, and as each sin, each error is thrust out, a
  fuller and clearer measure of the Light of Truth illumines the pilgrim
  soul.
  
  Thus meditating, you will be ceaselessly fortifying yourself against
  your only \emph{real} enemy, your selfish, perishable self, and will be
  establishing yourself more and more firmly in the divine and
  imperishable self that is inseparable from Truth. The direct outcome of
  your meditations will be a calm, spiritual strength which will be your
  stay and resting-place in the struggle of life. Great is the overcoming
  power of holy thought, and the strength and knowledge gained in the hour
  of silent meditation will enrich the soul with saving remembrance in the
  hour of strife, of sorrow, or of temptation.
  
  As, by the power of meditation, you grow in wisdom, you will relinquish,
  more and more, your selfish desires which are fickle, impermanent, and
  productive of sorrow and pain; and will take your stand, with increasing
  steadfastness and trust, upon unchangeable principles, and will realize
  heavenly rest.
  
  The use of meditation is the acquirement of a knowledge of eternal
  principles, and the power which results from meditation is the ability
  to rest upon and trust those principles, and so become one with the
  Eternal. The end of meditation is, therefore, direct knowledge of Truth,
  God, and the realization of divine and profound peace.
  
  Let your meditations take their rise from the ethical ground which you
  now occupy. Remember that you are to \emph{grow} into Truth by steady
  perseverance. If you are an orthodox Christian, meditate ceaselessly
  upon the spotless purity and divine excellence of the character of
  Jesus, and apply his every precept to your inner life and outward
  conduct, so as to approximate more and more toward his perfection. Do
  not be as those religious ones, who, refusing to meditate upon the Law
  of Truth, and to put into practice the precepts given to them by their
  Master, are content to formally worship, to cling to their particular
  creeds, and to continue in the ceaseless round of sin and suffering.
  Strive to rise, by the power of meditation, above all selfish clinging
  to partial gods or party creeds; above dead formalities and lifeless
  ignorance. Thus walking the high way of wisdom, with mind fixed upon the
  spotless Truth, you shall know no halting-place short of the realization
  of Truth.
  
  He who earnestly meditates first perceives a truth, as it were, afar
  off, and then realizes it by daily practice. It is only the doer of the
  Word of Truth that can know of the doctrine of Truth, for though by pure
  thought the Truth is perceived, it is only actualized by practice.
  
  Said the divine Gautama, the Buddha, ``He who gives himself up to
  vanity, and does not give himself up to meditation, forgetting the real
  aim of life and grasping at pleasure, will in time envy him who has
  exerted himself in meditation,'' and he instructed his disciples in the
  following ``Five Great Meditations'':--
  
  ``The first meditation is the meditation of love, in which you so adjust
  your heart that you long for the weal and welfare of all beings,
  including the happiness of your enemies.
  
  ``The second meditation is the meditation of pity, in which you think of
  all beings in distress, vividly representing in your imagination their
  sorrows and anxieties so as to arouse a deep compassion for them in your
  soul.
  
  ``The third meditation is the meditation of joy, in which you think of
  the prosperity of others, and rejoice with their rejoicings.
  
  ``The fourth meditation is the meditation of impurity, in which you
  consider the evil consequences of corruption, the effects of sin and
  diseases. How trivial often the pleasure of the moment, and how fatal
  its consequences.
  
  ``The fifth meditation is the meditation on serenity, in which you rise
  above love and hate, tyranny and oppression, wealth and want, and regard
  your own fate with impartial calmness and perfect tranquility.''
  
  By engaging in these meditations the disciples of the Buddha arrived at
  a knowledge of the Truth. But whether you engage in these particular
  meditations or not matters little so long as your object is Truth, so
  long as you hunger and thirst for that righteousness which is a holy
  heart and a blameless life. In your meditations, therefore, let your
  heart grow and expand with ever-broadening love, until, freed from all
  hatred, and passion, and condemnation, it embraces the whole universe
  with thoughtful tenderness. As the flower opens its petals to receive
  the morning light, so open your soul more and more to the glorious light
  of Truth. Soar upward upon the wings of aspiration; be fearless, and
  believe in the loftiest possibilities. Believe that a life of absolute
  meekness is possible; believe that a life of stainless purity is
  possible; believe that a life of perfect holiness is possible; believe
  that the realization of the highest truth is possible. He who so
  believes, climbs rapidly the heavenly hills, whilst the unbelievers
  continue to grope darkly and painfully in the fog-bound valleys.
  
  So believing, so aspiring, so meditating, divinely sweet and beautiful
  will be your spiritual experiences, and glorious the revelations that
  will enrapture your inward vision. As you realize the divine Love, the
  divine Justice, the divine Purity, the Perfect Law of Good, or God,
  great will be your bliss and deep your peace. Old things will pass away,
  and all things will become new. The veil of the material universe, so
  dense and impenetrable to the eye of error, so thin and gauzy to the eye
  of Truth, will be lifted and the spiritual universe will be revealed.
  Time will cease, and you will live only in Eternity. Change and
  mortality will no more cause you anxiety and sorrow, for you will become
  established in the unchangeable, and will dwell in the very heart of
  immortality.
  
  \begin{itemize}
  \itemsep1pt\parskip0pt\parsep0pt
  \item
    STAR OF WISDOM *
  \end{itemize}
  
  Star that of the birth of Vishnu, Birth of Krishna, Buddha, Jesus, Told
  the wise ones, Heavenward looking, Waiting, watching for thy gleaming In
  the darkness of the night-time, In the starless gloom of midnight;
  Shining Herald of the coming Of the kingdom of the righteous; Teller of
  the Mystic story Of the lowly birth of Godhead In the stable of the
  passions, In the manger of the mind-soul; Silent singer of the secret Of
  compassion deep and holy To the heart with sorrow burdened, To the soul
  with waiting weary:-- Star of all-surpassing brightness, Thou again dost
  deck the midnight; Thou again dost cheer the wise ones Watching in the
  creedal darkness, Weary of the endless battle With the grinding blades
  of error; Tired of lifeless, useless idols, Of the dead forms of
  religions; Spent with watching for thy shining; Thou hast ended their
  despairing; Thou hast lighted up their pathway; Thou hast brought again
  the old Truths To the hearts of all thy Watchers; To the souls of them
  that love thee Thou dost speak of Joy and Gladness, Of the peace that
  comes of Sorrow. Blessed are they that can see thee, Weary wanderers in
  the Night-time; Blessed they who feel the throbbing, In their bosoms
  feel the pulsing Of a deep Love stirred within them By the great power
  of thy shining. Let us learn thy lesson truly; Learn it faithfully and
  humbly; Learn it meekly, wisely, gladly, Ancient Star of holy Vishnu,
  Light of Krishna, Buddha, Jesus.
  
  \section{THE TWO MASTERS, SELF AND
  TRUTH}\label{the-two-masters-self-and-truth}
  
  Upon the battlefield of the human soul two masters are ever contending
  for the crown of supremacy, for the kingship and dominion of the heart;
  the master of self, called also the ``Prince of this world,'' and the
  master of Truth, called also the Father God. The master self is that
  rebellious one whose weapons are passion, pride, avarice, vanity,
  self-will, implements of darkness; the master Truth is that meek and
  lowly one whose weapons are gentleness, patience, purity, sacrifice,
  humility, love, instruments of Light.
  
  In every soul the battle is waged, and as a soldier cannot engage at
  once in two opposing armies, so every heart is enlisted either in the
  ranks of self or of Truth. There is no half-and-half course; ``There is
  self and there is Truth; where self is, Truth is not, where Truth is,
  self is not.'' Thus spake Buddha, the teacher of Truth, and Jesus, the
  manifested Christ, declared that ``No man can serve two masters; for
  either he will hate the one and love the other; or else he will hold to
  the one, and despise the other. Ye cannot serve God and Mammon.''
  
  Truth is so simple, so absolutely undeviating and uncompromising that it
  admits of no complexity, no turning, no qualification. Self is
  ingenious, crooked, and, governed by subtle and snaky desire, admits of
  endless turnings and qualifications, and the deluded worshipers of self
  vainly imagine that they can gratify every worldly desire, and at the
  same time possess the Truth. But the lovers of Truth worship Truth with
  the sacrifice of self, and ceaselessly guard themselves against
  worldliness and self-seeking.
  
  Do you seek to know and to realize Truth? Then you must be prepared to
  sacrifice, to renounce to the uttermost, for Truth in all its glory can
  only be perceived and known when the last vestige of self has
  disappeared.
  
  The eternal Christ declared that he who would be His disciple must
  ``deny himself daily.'' Are you willing to deny yourself, to give up
  your lusts, your prejudices, your opinions? If so, you may enter the
  narrow way of Truth, and find that peace from which the world is shut
  out. The absolute denial, the utter extinction, of self is the perfect
  state of Truth, and all religions and philosophies are but so many aids
  to this supreme attainment.
  
  Self is the denial of Truth. Truth is the denial of self. As you let
  self die, you will be reborn in Truth. As you cling to self, Truth will
  be hidden from you.
  
  Whilst you cling to self, your path will be beset with difficulties, and
  repeated pains, sorrows, and disappointments will be your lot. There are
  no difficulties in Truth, and coming to Truth, you will be freed from
  all sorrow and disappointment.
  
  Truth in itself is not hidden and dark. It is always revealed and is
  perfectly transparent. But the blind and wayward self cannot perceive
  it. The light of day is not hidden except to the blind, and the Light of
  Truth is not hidden except to those who are blinded by self.
  
  Truth is the one Reality in the universe, the inward Harmony, the
  perfect Justice, the eternal Love. Nothing can be added to it, nor taken
  from it. It does not depend upon any man, but all men depend upon it.
  You cannot perceive the beauty of Truth while you are looking out
  through the eyes of self. If you are vain, you will color everything
  with your own vanities. If lustful, your heart and mind will be so
  clouded with the smoke and flames of passion, that everything will
  appear distorted through them. If proud and opinionative, you will see
  nothing in the whole universe except the magnitude and importance of
  your own opinions.
  
  There is one quality which pre-eminently distinguishes the man of Truth
  from the man of self, and that is \emph{humility}. To be not only free
  from vanity, stubbornness and egotism, but to regard one's own opinions
  as of no value, this indeed is true humility.
  
  He who is immersed in self regards his own opinions as Truth, and the
  opinions of other men as error. But that humble Truth-lover who has
  learned to distinguish between opinion and Truth, regards all men with
  the eye of charity, and does not seek to defend his opinions against
  theirs, but sacrifices those opinions that he may love the more, that he
  may manifest the spirit of Truth, for Truth in its very nature is
  ineffable and can only be lived. He who has most of charity has most of
  Truth.
  
  Men engage in heated controversies, and foolishly imagine they are
  defending the Truth, when in reality they are merely defending their own
  petty interests and perishable opinions. The follower of self takes up
  arms against others. The follower of Truth takes up arms against
  himself. Truth, being unchangeable and eternal, is independent of your
  opinion and of mine. We may enter into it, or we may stay outside; but
  both our defense and our attack are superfluous, and are hurled back
  upon ourselves.
  
  Men, enslaved by self, passionate, proud, and condemnatory, believe
  their particular creed or religion to be the Truth, and all other
  religions to be error; and they proselytize with passionate ardor. There
  is but one religion, the religion of Truth. There is but one error, the
  error of self. Truth is not a formal belief; it is an unselfish, holy,
  and aspiring heart, and he who has Truth is at peace with all, and
  cherishes all with thoughts of love.
  
  You may easily know whether you are a child of Truth or a worshiper of
  self, if you will silently examine your mind, heart, and conduct. Do you
  harbor thoughts of suspicion, enmity, envy, lust, pride, or do you
  strenuously fight against these? If the former, you are chained to self,
  no matter what religion you may profess; if the latter, you are a
  candidate for Truth, even though outwardly you may profess no religion.
  Are you passionate, self-willed, ever seeking to gain your own ends,
  self-indulgent, and self-centered; or are you gentle, mild, unselfish,
  quit of every form of self-indulgence, and are ever ready to give up
  your own? If the former, self is your master; if the latter, Truth is
  the object of your affection. Do you strive for riches? Do you fight,
  with passion, for your party? Do you lust for power and leadership? Are
  you given to ostentation and self-praise? Or have you given up the love
  of riches? Have you relinquished all strife? Are you content to take the
  lowest place, and to be passed by unnoticed? And have you ceased to talk
  about yourself and to regard yourself with self-complacent pride? If the
  former, even though you may imagine you worship God, the god of your
  heart is self. If the latter, even though you may withhold your lips
  from worship, you are dwelling with the Most High.
  
  The signs by which the Truth-lover is known are unmistakable. Hear the
  Holy Krishna declare them, in Sir Edwin Arnold's beautiful rendering of
  the ``Bhagavad Gita'':--
  
  \begin{quote}
  ``Fearlessness, singleness of soul, the will Always to strive for
  wisdom; opened hand And governed appetites; and piety, And love of
  lonely study; humbleness, Uprightness, heed to injure nought which lives
  Truthfulness, slowness unto wrath, a mind That lightly letteth go what
  others prize; And equanimity, and charity Which spieth no man's faults;
  and tenderness Towards all that suffer; a contented heart, Fluttered by
  no desires; a bearing mild, Modest and grave, with manhood nobly mixed,
  With patience, fortitude and purity; An unrevengeful spirit, never given
  To rate itself too high--such be the signs, O Indian Prince! of him
  whose feet are set On that fair path which leads to heavenly birth!''
  \end{quote}
  
  When men, lost in the devious ways of error and self, have forgotten the
  ``heavenly birth,'' the state of holiness and Truth, they set up
  artificial standards by which to judge one another, and make acceptance
  of, and adherence to, their own particular theology, the test of Truth;
  and so men are divided one against another, and there is ceaseless
  enmity and strife, and unending sorrow and suffering.
  
  Reader, do you seek to realize the birth into Truth? There is only one
  way: \emph{Let self die}. All those lusts, appetites, desires, opinions,
  limited conceptions and prejudices to which you have hitherto so
  tenaciously clung, let them fall from you. Let them no longer hold you
  in bondage, and Truth will be yours. Cease to look upon your own
  religion as superior to all others, and strive humbly to learn the
  supreme lesson of charity. No longer cling to the idea, so productive of
  strife and sorrow, that the Savior whom you worship is the only Savior,
  and that the Savior whom your brother worships with equal sincerity and
  ardor, is an impostor; but seek diligently the path of holiness, and
  then you will realize that every holy man is a savior of mankind.
  
  The giving up of self is not merely the renunciation of outward things.
  It consists of the renunciation of the inward sin, the inward error. Not
  by giving up vain clothing; not by relinquishing riches; not by
  abstaining from certain foods; not by speaking smooth words; not by
  merely doing these things is the Truth found; but by giving up the
  spirit of vanity; by relinquishing the desire for riches; by abstaining
  from the lust of self-indulgence; by giving up all hatred, strife,
  condemnation, and self-seeking, and becoming gentle and pure at heart;
  by doing these things is the Truth found. To do the former, and not to
  do the latter, is pharisaism and hypocrisy, whereas the latter includes
  the former. You may renounce the outward world, and isolate yourself in
  a cave or in the depths of a forest, but you will take all your
  selfishness with you, and unless you renounce that, great indeed will be
  your wretchedness and deep your delusion. You may remain just where you
  are, performing all your duties, and yet renounce the world, the inward
  enemy. To be in the world and yet not of the world is the highest
  perfection, the most blessed peace, is to achieve the greatest victory.
  The renunciation of self is the way of Truth, therefore,
  
  \begin{quote}
  ``Enter the Path; there is no grief like hate, No pain like passion, no
  deceit like sense; Enter the Path; far hath he gone whose foot Treads
  down one fond offense.''
  \end{quote}
  
  As you succeed in overcoming self you will begin to see things in their
  right relations. He who is swayed by any passion, prejudice, like or
  dislike, adjusts everything to that particular bias, and sees only his
  own delusions. He who is absolutely free from all passion, prejudice,
  preference, and partiality, sees himself as he is; sees others as they
  are; sees all things in their proper proportions and right relations.
  Having nothing to attack, nothing to defend, nothing to conceal, and no
  interests to guard, he is at peace. He has realized the profound
  simplicity of Truth, for this unbiased, tranquil, blessed state of mind
  and heart is the state of Truth. He who attains to it dwells with the
  angels, and sits at the footstool of the Supreme. Knowing the Great Law;
  knowing the origin of sorrow; knowing the secret of suffering; knowing
  the way of emancipation in Truth, how can such a one engage in strife or
  condemnation; for though he knows that the blind, self-seeking world,
  surrounded with the clouds of its own illusions, and enveloped in the
  darkness of error and self, cannot perceive the steadfast Light of
  Truth, and is utterly incapable of comprehending the profound simplicity
  of the heart that has died, or is dying, to self, yet he also knows that
  when the suffering ages have piled up mountains of sorrow, the crushed
  and burdened soul of the world will fly to its final refuge, and that
  when the ages are completed, every prodigal will come back to the fold
  of Truth. And so he dwells in goodwill toward all, and regards all with
  that tender compassion which a father bestows upon his wayward children.
  
  Men cannot understand Truth because they cling to self, because they
  believe in and love self, because they believe self to be the only
  reality, whereas it is the one delusion.
  
  When you cease to believe in and love self you will desert it, and will
  fly to Truth, and will find the eternal Reality.
  
  When men are intoxicated with the wines of luxury, and pleasure, and
  vanity, the thirst of life grows and deepens within them, and they
  delude themselves with dreams of fleshly immortality, but when they come
  to reap the harvest of their own sowing, and pain and sorrow supervene,
  then, crushed and humiliated, relinquishing self and all the
  intoxications of self, they come, with aching hearts to the one
  immortality, the immortality that destroys all delusions, the spiritual
  immortality in Truth.
  
  Men pass from evil to good, from self to Truth, through the dark gate of
  sorrow, for sorrow and self are inseparable. Only in the peace and bliss
  of Truth is all sorrow vanquished. If you suffer disappointment because
  your cherished plans have been thwarted, or because someone has not come
  up to your anticipations, it is because you are clinging to self. If you
  suffer remorse for your conduct, it is because you have given way to
  self. If you are overwhelmed with chagrin and regret because of the
  attitude of someone else toward you, it is because you have been
  cherishing self. If you are wounded on account of what has been done to
  you or said of you, it is because you are walking in the painful way of
  self. All suffering is of self. All suffering ends in Truth. When you
  have entered into and realized Truth, you will no longer suffer
  disappointment, remorse, and regret, and sorrow will flee from you.
  
  \begin{quote}
  ``Self is the only prison that can ever bind the soul; Truth is the only
  angel that can bid the gates unroll; And when he comes to call thee,
  arise and follow fast; His way may lie through darkness, but it leads to
  light at last.''
  \end{quote}
  
  The woe of the world is of its own making. Sorrow purifies and deepens
  the soul, and the extremity of sorrow is the prelude to Truth.
  
  Have you suffered much? Have you sorrowed deeply? Have you pondered
  seriously upon the problem of life? If so, you are prepared to wage war
  against self, and to become a disciple of Truth.
  
  The intellectual who do not see the necessity for giving up self, frame
  endless theories about the universe, and call them Truth; but do thou
  pursue that direct line of conduct which is the practice of
  righteousness, and thou wilt realize the Truth which has no place in
  theory, and which never changes. Cultivate your heart. Water it
  continually with unselfish love and deep-felt pity, and strive to shut
  out from it all thoughts and feelings which are not in accordance with
  Love. Return good for evil, love for hatred, gentleness for
  ill-treatment, and remain silent when attacked. So shall you transmute
  all your selfish desires into the pure gold of Love, and self will
  disappear in Truth. So will you walk blamelessly among men, yoked with
  the easy yoke of lowliness, and clothed with the divine garment of
  humility.
  
  \begin{quote}
  O come, weary brother! thy struggling and striving End thou in the heart
  of the Master of ruth; Across self's drear desert why wilt thou be
  driving, Athirst for the quickening waters of Truth
  \end{quote}
  
  \begin{quote}
  When here, by the path of thy searching and sinning, Flows Life's
  gladsome stream, lies Love's oasis green? Come, turn thou and rest; know
  the end and beginning, The sought and the searcher, the seer and seen.
  \end{quote}
  
  \begin{quote}
  Thy Master sits not in the unapproached mountains, Nor dwells in the
  mirage which floats on the air, Nor shalt thou discover His magical
  fountains In pathways of sand that encircle despair.
  \end{quote}
  
  \begin{quote}
  In selfhood's dark desert cease wearily seeking The odorous tracks of
  the feet of thy King; And if thou wouldst hear the sweet sound of His
  speaking, Be deaf to all voices that emptily sing.
  \end{quote}
  
  \begin{quote}
  Flee the vanishing places; renounce all thou hast; Leave all that thou
  lovest, and, naked and bare, Thyself at the shrine of the
  \emph{Innermost} cast; The Highest, the Holiest, the Changeless is
  there.
  \end{quote}
  
  \begin{quote}
  Within, in the heart of the Silence He dwelleth; Leave sorrow and sin,
  leave thy wanderings sore; Come bathe in His Joy, whilst He, whispering,
  telleth Thy soul what it seeketh, and wander no more.
  \end{quote}
  
  \begin{quote}
  Then cease, weary brother, thy struggling and striving; Find peace in
  the heart of the Master of ruth. Across self's dark desert cease wearily
  driving; Come; drink at the beautiful waters of Truth.
  \end{quote}
  
  \section{THE ACQUIREMENT OF SPIRITUAL
  POWER}\label{the-acquirement-of-spiritual-power}
  
  The world is filled with men and women seeking pleasure, excitement,
  novelty; seeking ever to be moved to laughter or tears; not seeking
  strength, stability, and power; but courting weakness, and eagerly
  engaged in dispersing what power they have.
  
  Men and women of real power and influence are few, because few are
  prepared to make the sacrifice necessary to the acquirement of power,
  and fewer still are ready to patiently build up character.
  
  To be swayed by your fluctuating thoughts and impulses is to be weak and
  powerless; to rightly control and direct those forces is to be strong
  and powerful. Men of strong animal passions have much of the ferocity of
  the beast, but this is not power. The elements of power are there; but
  it is only when this ferocity is tamed and subdued by the higher
  intelligence that real power begins; and men can only grow in power by
  awakening themselves to higher and ever higher states of intelligence
  and consciousness.
  
  The difference between a man of weakness and one of power lies not in
  the strength of the personal will (for the stubborn man is usually weak
  and foolish), but in that focus of consciousness which represents their
  states of knowledge.
  
  The pleasure-seekers, the lovers of excitement, the hunters after
  novelty, and the victims of impulse and hysterical emotion lack that
  knowledge of principles which gives balance, stability, and influence.
  
  A man commences to develop power when, checking his impulses and selfish
  inclinations, he falls back upon the higher and calmer consciousness
  within him, and begins to steady himself upon a principle. The
  realization of unchanging principles in consciousness is at once the
  source and secret of the highest power.
  
  When, after much searching, and suffering, and sacrificing, the light of
  an eternal principle dawns upon the soul, a divine calm ensues and joy
  unspeakable gladdens the heart.
  
  He who has realized such a principle ceases to wander, and remains
  poised and self-possessed. He ceases to be ``passion's slave,'' and
  becomes a master-builder in the Temple of Destiny.
  
  The man that is governed by self, and not by a principle, changes his
  front when his selfish comforts are threatened. Deeply intent upon
  defending and guarding his own interests, he regards all means as lawful
  that will subserve that end. He is continually scheming as to how he may
  protect himself against his enemies, being too self-centered to perceive
  that he is his own enemy. Such a man's work crumbles away, for it is
  divorced from Truth and power. All effort that is grounded upon self,
  perishes; only that work endures that is built upon an indestructible
  principle.
  
  The man that stands upon a principle is the same calm, dauntless,
  self-possessed man under all circumstances. When the hour of trial
  comes, and he has to decide between his personal comforts and Truth, he
  gives up his comforts and remains firm. Even the prospect of torture and
  death cannot alter or deter him. The man of self regards the loss of his
  wealth, his comforts, or his life as the greatest calamities which can
  befall him. The man of principle looks upon these incidents as
  comparatively insignificant, and not to be weighed with loss of
  character, loss of Truth. To desert Truth is, to him, the only happening
  which can really be called a calamity.
  
  It is the hour of crisis which decides who are the minions of darkness,
  and who the children of Light. It is the epoch of threatening disaster,
  ruin, and persecution which divides the sheep from the goats, and
  reveals to the reverential gaze of succeeding ages the men and women of
  power.
  
  It is easy for a man, so long as he is left in the enjoyment of his
  possessions, to persuade himself that he believes in and adheres to the
  principles of Peace, Brotherhood, and Universal Love; but if, when his
  enjoyments are threatened, or he imagines they are threatened, he begins
  to clamor loudly for war, he shows that he believes in and stands upon,
  not Peace, Brotherhood, and Love, but strife, selfishness, and hatred.
  
  He who does not desert his principles when threatened with the loss of
  every earthly thing, even to the loss of reputation and life, is the man
  of power; is the man whose every word and work endures; is the man whom
  the afterworld honors, reveres, and worships. Rather than desert that
  principle of Divine Love on which he rested, and in which all his trust
  was placed, Jesus endured the utmost extremity of agony and deprivation;
  and today the world prostrates itself at his pierced feet in rapt
  adoration.
  
  There is no way to the acquirement of spiritual power except by that
  inward illumination and enlightenment which is the realization of
  spiritual principles; and those principles can only be realized by
  constant practice and application.
  
  Take the principle of divine Love, and quietly and diligently meditate
  upon it with the object of arriving at a thorough understanding of it.
  Bring its searching light to bear upon all your habits, your actions,
  your speech and intercourse with others, your every secret thought and
  desire. As you persevere in this course, the divine Love will become
  more and more perfectly revealed to you, and your own shortcomings will
  stand out in more and more vivid contrast, spurring you on to renewed
  endeavor; and having once caught a glimpse of the incomparable majesty
  of that imperishable principle, you will never again rest in your
  weakness, your selfishness, your imperfection, but will pursue that Love
  until you have relinquished every discordant element, and have brought
  yourself into perfect harmony with it. And that state of inward harmony
  is spiritual power. Take also other spiritual principles, such as Purity
  and Compassion, and apply them in the same way, and, so exacting is
  Truth, you will be able to make no stay, no resting-place until the
  inmost garment of your soul is bereft of every stain, and your heart has
  become incapable of any hard, condemnatory, and pitiless impulse.
  
  Only in so far as you understand, realize, and rely upon, these
  principles, will you acquire spiritual power, and that power will be
  manifested in and through you in the form of increasing dispassion,
  patience and equanimity.
  
  Dispassion argues superior self-control; sublime patience is the very
  hall-mark of divine knowledge, and to retain an unbroken calm amid all
  the duties and distractions of life, marks off the man of power. ``It is
  easy in the world to live after the world's opinion; it is easy in
  solitude to live after our own; but the great man is he who in the midst
  of the crowd keeps with perfect sweetness the independence of
  solitude.''
  
  Some mystics hold that perfection in dispassion is the source of that
  power by which miracles (so-called) are performed, and truly he who has
  gained such perfect control of all his interior forces that no shock,
  however great, can for one moment unbalance him, must be capable of
  guiding and directing those forces with a master-hand.
  
  To grow in self-control, in patience, in equanimity, is to grow in
  strength and power; and you can only thus grow by focusing your
  consciousness upon a principle. As a child, after making many and
  vigorous attempts to walk unaided, at last succeeds, after numerous
  falls, in accomplishing this, so you must enter the way of power by
  first attempting to stand alone. Break away from the tyranny of custom,
  tradition, conventionality, and the opinions of others, until you
  succeed in walking lonely and erect among men. Rely upon your own
  judgment; be true to your own conscience; follow the Light that is
  within you; all outward lights are so many will-o'-the-wisps. There will
  be those who will tell you that you are foolish; that your judgment is
  faulty; that your conscience is all awry, and that the Light within you
  is darkness; but heed them not. If what they say is true the sooner you,
  as a searcher for wisdom, find it out the better, and you can only make
  the discovery by bringing your powers to the test. Therefore, pursue
  your course bravely. Your conscience is at least your own, and to follow
  it is to be a man; to follow the conscience of another is to be a slave.
  You will have many falls, will suffer many wounds, will endure many
  buffetings for a time, but press on in faith, believing that sure and
  certain victory lies ahead. Search for a rock, a principle, and having
  found it cling to it; get it under your feet and stand erect upon it,
  until at last, immovably fixed upon it, you succeed in defying the fury
  of the waves and storms of selfishness.
  
  For selfishness in any and every form is dissipation, weakness, death;
  unselfishness in its spiritual aspect is conservation, power, life. As
  you grow in spiritual life, and become established upon principles, you
  will become as beautiful and as unchangeable as those principles, will
  taste of the sweetness of their immortal essence, and will realize the
  eternal and indestructible nature of the God within.
  
  \begin{quote}
  No harmful shaft can reach the righteous man, Standing erect amid the
  storms of hate, Defying hurt and injury and ban, Surrounded by the
  trembling slaves of Fate.
  \end{quote}
  
  \begin{quote}
  Majestic in the strength of silent power, Serene he stands, nor changes
  not nor turns; Patient and firm in suffering's darkest hour, Time bends
  to him, and death and doom he spurns.
  \end{quote}
  
  \begin{quote}
  Wrath's lurid lightnings round about him play, And hell's deep thunders
  roll about his head; Yet heeds he not, for him they cannot slay Who
  stands whence earth and time and space are fled.
  \end{quote}
  
  \begin{quote}
  Sheltered by deathless love, what fear hath he? Armored in changeless
  Truth, what can he know Of loss and gain? Knowing eternity, He moves not
  whilst the shadows come and go.
  \end{quote}
  
  \begin{quote}
  Call him immortal, call him Truth and Light And splendor of prophetic
  majesty Who bideth thus amid the powers of night, Clothed with the glory
  of divinity.
  \end{quote}
  
  \section{THE REALIZATION OF SELFLESS
  LOVE}\label{the-realization-of-selfless-love}
  
  It is said that Michael Angelo saw in every rough block of stone a thing
  of beauty awaiting the master-hand to bring it into reality. Even so,
  within each there reposes the Divine Image awaiting the master-hand of
  Faith and the chisel of Patience to bring it into manifestation. And
  that Divine Image is revealed and realized as stainless, selfless Love.
  
  Hidden deep in every human heart, though frequently covered up with a
  mass of hard and almost impenetrable accretions, is the spirit of Divine
  Love, whose holy and spotless essence is undying and eternal. It is the
  Truth in man; it is that which belongs to the Supreme: that which is
  real and immortal. All else changes and passes away; this alone is
  permanent and imperishable; and to realize this Love by ceaseless
  diligence in the practice of the highest righteousness, to live in it
  and to become fully conscious in it, is to enter into immortality here
  and now, is to become one with Truth, one with God, one with the central
  Heart of all things, and to know our own divine and eternal nature.
  
  To reach this Love, to understand and experience it, one must work with
  great persistency and diligence upon his heart and mind, must ever renew
  his patience and keep strong his faith, for there will be much to
  remove, much to accomplish before the Divine Image is revealed in all
  its glorious beauty.
  
  He who strives to reach and to accomplish the divine will be tried to
  the very uttermost; and this is absolutely necessary, for how else could
  one acquire that sublime patience without which there is no real wisdom,
  no divinity? Ever and anon, as he proceeds, all his work will seem to be
  futile, and his efforts appear to be thrown away. Now and then a hasty
  touch will mar his image, and perhaps when he imagines his work is
  almost completed he will find what he imagined to be the beautiful form
  of Divine Love utterly destroyed, and he must begin again with his past
  bitter experience to guide and help him. But he who has resolutely set
  himself to realize the Highest recognizes no such thing as defeat. All
  failures are apparent, not real. Every slip, every fall, every return to
  selfishness is a lesson learned, an experience gained, from which a
  golden grain of wisdom is extracted, helping the striver toward the
  accomplishment of his lofty object. To recognize
  
  \begin{quote}
  ``That of our vices we can frame A ladder if we will but tread Beneath
  our feet each deed of shame,''
  \end{quote}
  
  is to enter the way that leads unmistakably toward the Divine, and the
  failings of one who thus recognizes are so many dead selves, upon which
  he rises, as upon stepping-stones, to higher things.
  
  Once come to regard your failings, your sorrows and sufferings as so
  many voices telling you plainly where you are weak and faulty, where you
  fall below the true and the divine, you will then begin to ceaselessly
  watch yourself, and every slip, every pang of pain will show you where
  you are to set to work, and what you have to remove out of your heart in
  order to bring it nearer to the likeness of the Divine, nearer to the
  Perfect Love. And as you proceed, day by day detaching yourself more and
  more from the inward selfishness the Love that is selfless will
  gradually become revealed to you. And when you are growing patient and
  calm, when your petulances, tempers, and irritabilities are passing away
  from you, and the more powerful lusts and prejudices cease to dominate
  and enslave you, then you will know that the divine is awakening within
  you, that you are drawing near to the eternal Heart, that you are not
  far from that selfless Love, the possession of which is peace and
  immortality.
  
  Divine Love is distinguished from human loves in this supremely
  important particular, \emph{it is free from partiality}. Human loves
  cling to a particular object to the exclusion of all else, and when that
  object is removed, great and deep is the resultant suffering to the one
  who loves. Divine Love embraces the whole universe, and, without
  clinging to any part, yet contains within itself the whole, and he who
  comes to it by gradually purifying and broadening his human loves until
  all the selfish and impure elements are burnt out of them, ceases from
  suffering. It is because human loves are narrow and confined and mingled
  with selfishness that they cause suffering. No suffering can result from
  that Love which is so absolutely pure that it seeks nothing for itself.
  Nevertheless, human loves are absolutely necessary as steps toward the
  Divine, and no soul is prepared to partake of Divine Love until it has
  become capable of the deepest and most intense human love. It is only by
  passing through human loves and human sufferings that Divine Love is
  reached and realized.
  
  All human loves are perishable like the forms to which they cling; but
  there is a Love that is imperishable, and that does not cling to
  appearances.
  
  All human loves are counterbalanced by human hates; but there is a Love
  that admits of no opposite or reaction; divine and free from all taint
  of self, that sheds its fragrance on all alike.
  
  Human loves are reflections of the Divine Love, and draw the soul nearer
  to the reality, the Love that knows neither sorrow nor change.
  
  It is well that the mother, clinging with passionate tenderness to the
  little helpless form of flesh that lies on her bosom, should be
  overwhelmed with the dark waters of sorrow when she sees it laid in the
  cold earth. It is well that her tears should flow and her heart ache,
  for only thus can she be reminded of the evanescent nature of the joys
  and objects of sense, and be drawn nearer to the eternal and
  imperishable Reality.
  
  It is well that lover, brother, sister, husband, wife should suffer deep
  anguish, and be enveloped in gloom when the visible object of their
  affections is torn from them, so that they may learn to turn their
  affections toward the invisible Source of all, where alone abiding
  satisfaction is to be found.
  
  It is well that the proud, the ambitious, the self-seeking, should
  suffer defeat, humiliation, and misfortune; that they should pass
  through the scorching fires of affliction; for only thus can the wayward
  soul be brought to reflect upon the enigma of life; only thus can the
  heart be softened and purified, and prepared to receive the Truth.
  
  When the sting of anguish penetrates the heart of human love; when gloom
  and loneliness and desertion cloud the soul of friendship and trust,
  then it is that the heart turns toward the sheltering love of the
  Eternal, and finds rest in its silent peace. And whosoever comes to this
  Love is not turned away comfortless, is not pierced with anguish nor
  surrounded with gloom; and is never deserted in the dark hour of trial.
  
  The glory of Divine Love can only be revealed in the heart that is
  chastened by sorrow, and the image of the heavenly state can only be
  perceived and realized when the lifeless, formless accretions of
  ignorance and self are hewn away.
  
  Only that Love that seeks no personal gratification or reward, that does
  not make distinctions, and that leaves behind no heartaches, can be
  called divine.
  
  Men, clinging to self and to the comfortless shadows of evil, are in the
  habit of thinking of divine Love as something belonging to a God who is
  out of reach; as something outside themselves, and that must for ever
  remain outside. Truly, the Love of God is ever beyond the reach of self,
  but when the heart and mind are emptied of self then the selfless Love,
  the supreme Love, the Love that is of God or Good becomes an inward and
  abiding reality.
  
  And this inward realization of holy Love is none other than the Love of
  Christ that is so much talked about and so little comprehended. The Love
  that not only saves the soul from sin, but lifts it also above the power
  of temptation.
  
  But how may one attain to this sublime realization? The answer which
  Truth has always given, and will ever give to this question is,--``Empty
  thyself, and I will fill thee.'' Divine Love cannot be known until self
  is dead, for self is the denial of Love, and how can that which is known
  be also denied? Not until the stone of self is rolled away from the
  sepulcher of the soul does the immortal Christ, the pure Spirit of Love,
  hitherto crucified, dead and buried, cast off the bands of ignorance,
  and come forth in all the majesty of His resurrection.
  
  You believe that the Christ of Nazareth was put to death and rose again.
  I do not say you err in that belief; but if you refuse to believe that
  the gentle spirit of Love is crucified daily upon the dark cross of your
  selfish desires, then, I say, you err in this unbelief, and have not yet
  perceived, even afar off, the Love of Christ.
  
  You say that you have tasted of salvation in the Love of Christ. Are you
  saved from your temper, your irritability, your vanity, your personal
  dislikes, your judgment and condemnation of others? If not, from what
  are you saved, and wherein have you realized the transforming Love of
  Christ?
  
  He who has realized the Love that is divine has become a new man, and
  has ceased to be swayed and dominated by the old elements of self. He is
  known for his patience, his purity, his self-control, his deep charity
  of heart, and his unalterable sweetness.
  
  Divine or selfless Love is not a mere sentiment or emotion; it is a
  state of knowledge which destroys the dominion of evil and the belief in
  evil, and lifts the soul into the joyful realization of the supreme
  Good. To the divinely wise, knowledge and Love are one and inseparable.
  
  It is toward the complete realization of this divine Love that the whole
  world is moving; it was for this purpose that the universe came into
  existence, and every grasping at happiness, every reaching out of the
  soul toward objects, ideas and ideals, is an effort to realize it. But
  the world does not realize this Love at present because it is grasping
  at the fleeting shadow and ignoring, in its blindness, the substance.
  And so suffering and sorrow continue, and must continue until the world,
  taught by its self-inflicted pains, discovers the Love that is selfless,
  the wisdom that is calm and full of peace.
  
  And this Love, this Wisdom, this Peace, this tranquil state of mind and
  heart may be attained to, may be realized by all who are willing and
  ready to yield up self, and who are prepared to humbly enter into a
  comprehension of all that the giving up of self involves. There is no
  arbitrary power in the universe, and the strongest chains of fate by
  which men are bound are self-forged. Men are chained to that which
  causes suffering because they desire to be so, because they love their
  chains, because they think their little dark prison of self is sweet and
  beautiful, and they are afraid that if they desert that prison they will
  lose all that is real and worth having.
  
  \begin{quote}
  ``Ye suffer from yourselves, none else compels, None other holds ye that
  ye live and die.''
  \end{quote}
  
  And the indwelling power which forged the chains and built around itself
  the dark and narrow prison, can break away when it desires and wills to
  do so, and the soul does will to do so when it has discovered the
  worthlessness of its prison, when long suffering has prepared it for the
  reception of the boundless Light and Love.
  
  As the shadow follows the form, and as smoke comes after fire, so effect
  follows cause, and suffering and bliss follow the thoughts and deeds of
  men. There is no effect in the world around us but has its hidden or
  revealed cause, and that cause is in accordance with absolute justice.
  Men reap a harvest of suffering because in the near or distant past they
  have sown the seeds of evil; they reap a harvest of bliss also as a
  result of their own sowing of the seeds of good. Let a man meditate upon
  this, let him strive to understand it, and he will then begin to sow
  only seeds of good, and will burn up the tares and weeds which he has
  formerly grown in the garden of his heart.
  
  The world does not understand the Love that is selfless because it is
  engrossed in the pursuit of its own pleasures, and cramped within the
  narrow limits of perishable interests mistaking, in its ignorance, those
  pleasures and interests for real and abiding things. Caught in the
  flames of fleshly lusts, and burning with anguish, it sees not the pure
  and peaceful beauty of Truth. Feeding upon the swinish husks of error
  and self-delusion, it is shut out from the mansion of all-seeing Love.
  
  Not having this Love, not understanding it, men institute innumerable
  reforms which involve no inward sacrifice, and each imagines that his
  reform is going to right the world for ever, while he himself continues
  to propagate evil by engaging it in his own heart. That only can be
  called reform which tends to reform the human heart, for all evil has
  its rise there, and not until the world, ceasing from selfishness and
  party strife, has learned the lesson of divine Love, will it realize the
  Golden Age of universal blessedness.
  
  Let the rich cease to despise the poor, and the poor to condemn the
  rich; let the greedy learn how to give, and the lustful how to grow
  pure; let the partisan cease from strife, and the uncharitable begin to
  forgive; let the envious endeavor to rejoice with others, and the
  slanderers grow ashamed of their conduct. Let men and women take this
  course, and, lo! the Golden Age is at hand. He, therefore, who purifies
  his own heart is the world's greatest benefactor.
  
  Yet, though the world is, and will be for many ages to come, shut out
  from that Age of Gold, which is the realization of selfless Love, you,
  if you are willing, may enter it now, by rising above your selfish self;
  if you will pass from prejudice, hatred, and condemnation, to gentle and
  forgiving love.
  
  Where hatred, dislike, and condemnation are, selfless Love does not
  abide. It resides only in the heart that has ceased from all
  condemnation.
  
  You say, ``How can I love the drunkard, the hypocrite, the sneak, the
  murderer? I am compelled to dislike and condemn such men.'' It is true
  you cannot love such men \emph{emotionally}, but when you say that you
  must perforce dislike and condemn them you show that you are not
  acquainted with the Great over-ruling Love; for it is possible to attain
  to such a state of interior enlightenment as will enable you to perceive
  the train of causes by which these men have become as they are, to enter
  into their intense sufferings, and to know the certainty of their
  ultimate purification. Possessed of such knowledge it will be utterly
  impossible for you any longer to dislike or condemn them, and you will
  always think of them with perfect calmness and deep compassion.
  
  If you love people and speak of them with praise until they in some way
  thwart you, or do something of which you disapprove, and then you
  dislike them and speak of them with dispraise, you are not governed by
  the Love which is of God. If, in your heart, you are continually
  arraigning and condemning others, selfless Love is hidden from you.
  
  He who knows that Love is at the heart of all things, and has realized
  the all-sufficing power of that Love, has no room in his heart for
  condemnation.
  
  Men, not knowing this Love, constitute themselves judge and executioner
  of their fellows, forgetting that there is the Eternal Judge and
  Executioner, and in so far as men deviate from them in their own views,
  their particular reforms and methods, they brand them as fanatical,
  unbalanced, lacking judgment, sincerity, and honesty; in so far as
  others approximate to their own standard do they look upon them as being
  everything that is admirable. Such are the men who are centered in self.
  But he whose heart is centered in the supreme Love does not so brand and
  classify men; does not seek to convert men to his own views, not to
  convince them of the superiority of his methods. Knowing the Law of
  Love, he lives it, and maintains the same calm attitude of mind and
  sweetness of heart toward all. The debased and the virtuous, the foolish
  and the wise, the learned and the unlearned, the selfish and the
  unselfish receive alike the benediction of his tranquil thought.
  
  You can only attain to this supreme knowledge, this divine Love by
  unremitting endeavor in self-discipline, and by gaining victory after
  victory over yourself. Only the pure in heart see God, and when your
  heart is sufficiently purified you will enter into the New Birth, and
  the Love that does not die, nor change, nor end in pain and sorrow will
  be awakened within you, and you will be at peace.
  
  He who strives for the attainment of divine Love is ever seeking to
  overcome the spirit of condemnation, for where there is pure spiritual
  knowledge, condemnation cannot exist, and only in the heart that has
  become incapable of condemnation is Love perfected and fully realized.
  
  The Christian condemns the Atheist; the Atheist satirizes the Christian;
  the Catholic and Protestant are ceaselessly engaged in wordy warfare,
  and the spirit of strife and hatred rules where peace and love should
  be.
  
  ``He that hateth his brother is a murderer,'' a crucifier of the divine
  Spirit of Love; and until you can regard men of all religions and of no
  religion with the same impartial spirit, with all freedom from dislike,
  and with perfect equanimity, you have yet to strive for that Love which
  bestows upon its possessor freedom and salvation.
  
  The realization of divine knowledge, selfless Love, utterly destroys the
  spirit of condemnation, disperses all evil, and lifts the consciousness
  to that height of pure vision where Love, Goodness, Justice are seen to
  be universal, supreme, all-conquering, indestructible.
  
  Train your mind in strong, impartial, and gentle thought; train your
  heart in purity and compassion; train your tongue to silence and to true
  and stainless speech; so shall you enter the way of holiness and peace,
  and shall ultimately realize the immortal Love. So living, without
  seeking to convert, you will convince; without arguing, you will teach;
  not cherishing ambition, the wise will find you out; and without
  striving to gain men's opinions, you will subdue their hearts. For Love
  is all-conquering, all-powerful; and the thoughts, and deeds, and words
  of Love can never perish.
  
  To know that Love is universal, supreme, all-sufficing; to be freed from
  the trammels of evil; to be quit of the inward unrest; to know that all
  men are striving to realize the Truth each in his own way; to be
  satisfied, sorrowless, serene; this is peace; this is gladness; this is
  immortality; this is Divinity; this is the realization of selfless Love.
  
  \begin{quote}
  I stood upon the shore, and saw the rocks Resist the onslaught of the
  mighty sea, And when I thought how all the countless shocks They had
  withstood through an eternity, I said, ``To wear away this solid main
  The ceaseless efforts of the waves are vain.''
  \end{quote}
  
  \begin{quote}
  But when I thought how they the rocks had rent, And saw the sand and
  shingles at my feet (Poor passive remnants of resistance spent) Tumbled
  and tossed where they the waters meet, Then saw I ancient landmarks
  'neath the waves, And knew the waters held the stones their slaves.
  \end{quote}
  
  \begin{quote}
  I saw the mighty work the waters wrought By patient softness and
  unceasing flow; How they the proudest promontory brought Unto their
  feet, and massy hills laid low; How the soft drops the adamantine wall
  Conquered at last, and brought it to its fall.
  \end{quote}
  
  \begin{quote}
  And then I knew that hard, resisting sin Should yield at last to Love's
  soft ceaseless roll Coming and going, ever flowing in Upon the proud
  rocks of the human soul; That all resistance should be spent and past,
  And every heart yield unto it at last.
  \end{quote}
  
  \section{ENTERING INTO THE INFINITE}\label{entering-into-the-infinite}
  
  From the beginning of time, man, in spite of his bodily appetites and
  desires, in the midst of all his clinging to earthly and impermanent
  things, has ever been intuitively conscious of the limited, transient,
  and illusionary nature of his material existence, and in his sane and
  silent moments has tried to reach out into a comprehension of the
  Infinite, and has turned with tearful aspiration toward the restful
  Reality of the Eternal Heart.
  
  While vainly imagining that the pleasures of earth are real and
  satisfying, pain and sorrow continually remind him of their unreal and
  unsatisfying nature. Ever striving to believe that complete satisfaction
  is to be found in material things, he is conscious of an inward and
  persistent revolt against this belief, which revolt is at once a
  refutation of his essential mortality, and an inherent and imperishable
  proof that only in the immortal, the eternal, the infinite can he find
  abiding satisfaction and unbroken peace.
  
  And here is the common ground of faith; here the root and spring of all
  religion; here the soul of Brotherhood and the heart of Love,--that man
  is essentially and spiritually divine and eternal, and that, immersed in
  mortality and troubled with unrest, he is ever striving to enter into a
  consciousness of his real nature.
  
  The spirit of man is inseparable from the Infinite, and can be satisfied
  with nothing short of the Infinite, and the burden of pain will continue
  to weigh upon man's heart, and the shadows of sorrow to darken his
  pathway until, ceasing from his wanderings in the dream-world of matter,
  he comes back to his home in the reality of the Eternal.
  
  As the smallest drop of water detached from the ocean contains all the
  qualities of the ocean, so man, detached in consciousness from the
  Infinite, contains within him its likeness; and as the drop of water
  must, by the law of its nature, ultimately find its way back to the
  ocean and lose itself in its silent depths, so must man, by the
  unfailing law of his nature, at last return to his source, and lose
  himself in the great ocean of the Infinite.
  
  To re-become one with the Infinite is the goal of man. To enter into
  perfect harmony with the Eternal Law is Wisdom, Love and Peace. But this
  divine state is, and must ever be, incomprehensible to the merely
  personal. Personality, separateness, selfishness are one and the same,
  and are the antithesis of wisdom and divinity. By the unqualified
  surrender of the personality, separateness and selfishness cease, and
  man enters into the possession of his divine heritage of immortality and
  infinity.
  
  Such surrender of the personality is regarded by the worldly and selfish
  mind as the most grievous of all calamities, the most irreparable loss,
  yet it is the one supreme and incomparable blessing, the only real and
  lasting gain. The mind unenlightened upon the inner laws of being, and
  upon the nature and destiny of its own life, clings to transient
  appearances, things which have in them no enduring substantiality, and
  so clinging, perishes, for the time being, amid the shattered wreckage
  of its own illusions.
  
  Men cling to and gratify the flesh as though it were going to last for
  ever, and though they try to forget the nearness and inevitability of
  its dissolution, the dread of death and of the loss of all that they
  cling to clouds their happiest hours, and the chilling shadow of their
  own selfishness follows them like a remorseless specter.
  
  And with the accumulation of temporal comforts and luxuries, the
  divinity within men is drugged, and they sink deeper and deeper into
  materiality, into the perishable life of the senses, and where there is
  sufficient intellect, theories concerning the immortality of the flesh
  come to be regarded as infallible truths. When a man's soul is clouded
  with selfishness in any or every form, he loses the power of spiritual
  discrimination, and confuses the temporal with the eternal, the
  perishable with the permanent, mortality with immortality, and error
  with Truth. It is thus that the world has come to be filled with
  theories and speculations having no foundation in human experience.
  Every body of flesh contains within itself, from the hour of birth, the
  elements of its own destruction, and by the unalterable law of its own
  nature must it pass away.
  
  The perishable in the universe can never become permanent; the permanent
  can never pass away; the mortal can never become immortal; the immortal
  can never die; the temporal cannot become eternal nor the eternal become
  temporal; appearance can never become reality, nor reality fade into
  appearance; error can never become Truth, nor can Truth become error.
  Man cannot immortalize the flesh, but, by overcoming the flesh, by
  relinquishing all its inclinations, he can enter the region of
  immortality. ``God alone hath immortality,'' and only by realizing the
  God state of consciousness does man enter into immortality.
  
  All nature in its myriad forms of life is changeable, impermanent,
  unenduring. Only the informing Principle of nature endures. Nature is
  many, and is marked by separation. The informing Principle is One, and
  is marked by unity. By overcoming the senses and the selfishness within,
  which is the overcoming of nature, man emerges from the chrysalis of the
  personal and illusory, and wings himself into the glorious light of the
  impersonal, the region of universal Truth, out of which all perishable
  forms come.
  
  Let men, therefore, practice self-denial; let them conquer their animal
  inclinations; let them refuse to be enslaved by luxury and pleasure; let
  them practice virtue, and grow daily into high and ever higher virtue,
  until at last they grow into the Divine, and enter into both the
  practice and the comprehension of humility, meekness, forgiveness,
  compassion, and love, which practice and comprehension constitute
  Divinity.
  
  ``Good-will gives insight,'' and only he who has so conquered his
  personality that he has but one attitude of mind, that of good-will,
  toward all creatures, is possessed of divine insight, and is capable of
  distinguishing the true from the false. The supremely good man is,
  therefore, the wise man, the divine man, the enlightened seer, the
  knower of the Eternal. Where you find unbroken gentleness, enduring
  patience, sublime lowliness, graciousness of speech, self-control,
  self-forgetfulness, and deep and abounding sympathy, look there for the
  highest wisdom, seek the company of such a one, for he has realized the
  Divine, he lives with the Eternal, he has become one with the Infinite.
  Believe not him that is impatient, given to anger, boastful, who clings
  to pleasure and refuses to renounce his selfish gratifications, and who
  practices not good-will and far-reaching compassion, for such a one hath
  not wisdom, vain is all his knowledge, and his works and words will
  perish, for they are grounded on that which passes away.
  
  Let a man abandon self, let him overcome the world, let him deny the
  personal; by this pathway only can he enter into the heart of the
  Infinite.
  
  The world, the body, the personality are mirages upon the desert of
  time; transitory dreams in the dark night of spiritual slumber, and
  those who have crossed the desert, those who are spiritually awakened,
  have alone comprehended the Universal Reality where all appearances are
  dispersed and dreaming and delusion are destroyed.
  
  There is one Great Law which exacts unconditional obedience, one
  unifying principle which is the basis of all diversity, one eternal
  Truth wherein all the problems of earth pass away like shadows. To
  realize this Law, this Unity, this Truth, is to enter into the Infinite,
  is to become one with the Eternal.
  
  To center one's life in the Great Law of Love is to enter into rest,
  harmony, peace. To refrain from all participation in evil and discord;
  to cease from all resistance to evil, and from the omission of that
  which is good, and to fall back upon unswerving obedience to the holy
  calm within, is to enter into the inmost heart of things, is to attain
  to a living, conscious experience of that eternal and infinite principle
  which must ever remain a hidden mystery to the merely perceptive
  intellect. Until this principle is realized, the soul is not established
  in peace, and he who so realizes is truly wise; not wise with the wisdom
  of the learned, but with the simplicity of a blameless heart and of a
  divine manhood.
  
  To enter into a realization of the Infinite and Eternal is to rise
  superior to time, and the world, and the body, which comprise the
  kingdom of darkness; and is to become established in immortality,
  Heaven, and the Spirit, which make up the Empire of Light.
  
  Entering into the Infinite is not a mere theory or sentiment. It is a
  vital experience which is the result of assiduous practice in inward
  purification. When the body is no longer believed to be, even remotely,
  the real man; when all appetites and desires are thoroughly subdued and
  purified; when the emotions are rested and calm, and when the
  oscillation of the intellect ceases and perfect poise is secured, then,
  and not till then, does consciousness become one with the Infinite; not
  until then is childlike wisdom and profound peace secured.
  
  Men grow weary and gray over the dark problems of life, and finally pass
  away and leave them unsolved because they cannot see their way out of
  the darkness of the personality, being too much engrossed in its
  limitations. Seeking to save his personal life, man forfeits the greater
  impersonal Life in Truth; clinging to the perishable, he is shut out
  from a knowledge of the Eternal.
  
  By the surrender of self all difficulties are overcome, and there is no
  error in the universe but the fire of inward sacrifice will burn it up
  like chaff; no problem, however great, but will disappear like a shadow
  under the searching light of self-abnegation. Problems exist only in our
  own self-created illusions, and they vanish away when self is yielded
  up. Self and error are synonymous. Error is involved in the darkness of
  unfathomable complexity, but eternal simplicity is the glory of Truth.
  
  Love of self shuts men out from Truth, and seeking their own personal
  happiness they lose the deeper, purer, and more abiding bliss. Says
  Carlyle--``There is in man a higher than love of happiness. He can do
  without happiness, and instead thereof find blessedness.
  
  \ldots{} Love not pleasure, love God. This is the Everlasting Yea,
  wherein all contradiction is solved; wherein whoso walks and works, it
  is well with him."
  
  He who has yielded up that self, that personality that men most love,
  and to which they cling with such fierce tenacity, has left behind him
  all perplexity, and has entered into a simplicity so profoundly simple
  as to be looked upon by the world, involved as it is in a network of
  error, as foolishness. Yet such a one has realized the highest wisdom,
  and is at rest in the Infinite. He ``accomplishes without striving,''
  and all problems melt before him, for he has entered the region of
  reality, and deals, not with changing effects, but with the unchanging
  principles of things. He is enlightened with a wisdom which is as
  superior to ratiocination, as reason is to animality. Having yielded up
  his lusts, his errors, his opinions and prejudices, he has entered into
  possession of the knowledge of God, having slain the selfish desire for
  heaven, and along with it the ignorant fear of hell; having relinquished
  even the love of life itself, he has gained supreme bliss and Life
  Eternal, the Life which bridges life and death, and knows its own
  immortality. Having yielded up all without reservation, he has gained
  all, and rests in peace on the bosom of the Infinite.
  
  Only he who has become so free from self as to be equally content to be
  annihilated as to live, or to live as to be annihilated, is fit to enter
  into the Infinite. Only he who, ceasing to trust his perishable self,
  has learned to trust in boundless measure the Great Law, the Supreme
  Good, is prepared to partake of undying bliss.
  
  For such a one there is no more regret, nor disappointment, nor remorse,
  for where all selfishness has ceased these sufferings cannot be; and
  whatever happens to him he knows that it is for his own good, and he is
  content, being no longer the servant of self, but the servant of the
  Supreme. He is no longer affected by the changes of earth, and when he
  hears of wars and rumors of wars his peace is not disturbed, and where
  men grow angry and cynical and quarrelsome, he bestows compassion and
  love. Though appearances may contradict it, he knows that the world is
  progressing, and that
  
  \begin{quote}
  ``Through its laughing and its weeping, Through its living and its
  keeping, Through its follies and its labors, weaving in and out of
  sight, To the end from the beginning, Through all virtue and all
  sinning, Reeled from God's great spool of Progress, runs the golden
  thread of light.''
  \end{quote}
  
  When a fierce storm is raging none are angered about it, because they
  know it will quickly pass away, and when the storms of contention are
  devastating the world, the wise man, looking with the eye of Truth and
  pity, knows that it will pass away, and that out of the wreckage of
  broken hearts which it leaves behind the immortal Temple of Wisdom will
  be built.
  
  Sublimely patient; infinitely compassionate; deep, silent, and pure, his
  very presence is a benediction; and when he speaks men ponder his words
  in their hearts, and by them rise to higher levels of attainment. Such
  is he who has entered into the Infinite, who by the power of utmost
  sacrifice has solved the sacred mystery of life.
  
  \begin{quote}
  Questioning Life and Destiny and Truth, I sought the dark and
  labyrinthine Sphinx, Who spake to me this strange and wondrous thing:--
  ``Concealment only lies in blinded eyes, And God alone can see the Form
  of God.''
  \end{quote}
  
  \begin{quote}
  I sought to solve this hidden mystery Vainly by paths of blindness and
  of pain, But when I found the Way of Love and Peace, Concealment ceased,
  and I was blind no more: Then saw I God e'en with the eyes of God.
  \end{quote}
  
  \section{SAINTS, SAGES, AND SAVIORS: THE LAW OF
  SERVICE}\label{saints-sages-and-saviors-the-law-of-service}
  
  The spirit of Love which is manifested as a perfect and rounded life, is
  the crown of being and the supreme end of knowledge upon this earth.
  
  The measure of a man's truth is the measure of his love, and Truth is
  far removed from him whose life is not governed by Love. The intolerant
  and condemnatory, even though they profess the highest religion, have
  the smallest measure of Truth; while those who exercise patience, and
  who listen calmly and dispassionately to all sides, and both arrive
  themselves at, and incline others to, thoughtful and unbiased
  conclusions upon all problems and issues, have Truth in fullest measure.
  The final test of wisdom is this,--how does a man live? What spirit does
  he manifest? How does he act under trial and temptation? Many men boast
  of being in possession of Truth who are continually swayed by grief,
  disappointment, and passion, and who sink under the first little trial
  that comes along. Truth is nothing if not unchangeable, and in so far as
  a man takes his stand upon Truth does he become steadfast in virtue,
  does he rise superior to his passions and emotions and changeable
  personality.
  
  Men formulate perishable dogmas, and call them Truth. Truth cannot be
  formulated; it is ineffable, and ever beyond the reach of intellect. It
  can only be experienced by practice; it can only be manifested as a
  stainless heart and a perfect life.
  
  Who, then, in the midst of the ceaseless pandemonium of schools and
  creeds and parties, has the Truth? He who lives it. He who practices it.
  He who, having risen above that pandemonium by overcoming himself, no
  longer engages in it, but sits apart, quiet, subdued, calm, and
  self-possessed, freed from all strife, all bias, all condemnation, and
  bestows upon all the glad and unselfish love of the divinity within him.
  
  He who is patient, calm, gentle, and forgiving under all circumstances,
  manifests the Truth. Truth will never be proved by wordy arguments and
  learned treatises, for if men do not perceive the Truth in infinite
  patience, undying forgiveness, and all-embracing compassion, no words
  can ever prove it to them.
  
  It is an easy matter for the passionate to be calm and patient when they
  are alone, or are in the midst of calmness. It is equally easy for the
  uncharitable to be gentle and kind when they are dealt kindly with, but
  he who retains his patience and calmness under all trial, who remains
  sublimely meek and gentle under the most trying circumstances, he, and
  he alone, is possessed of the spotless Truth. And this is so because
  such lofty virtues belong to the Divine, and can only be manifested by
  one who has attained to the highest wisdom, who has relinquished his
  passionate and self-seeking nature, who has realized the supreme and
  unchangeable Law, and has brought himself into harmony with it.
  
  Let men, therefore, cease from vain and passionate arguments about
  Truth, and let them think and say and do those things which make for
  harmony, peace, love, and good-will. Let them practice heart-virtue, and
  search humbly and diligently for the Truth which frees the soul from all
  error and sin, from all that blights the human heart, and that darkens,
  as with unending night, the pathway of the wandering souls of earth.
  
  There is one great all-embracing Law which is the foundation and cause
  of the universe, the Law of Love. It has been called by many names in
  various countries and at various times, but behind all its names the
  same unalterable Law may be discovered by the eye of Truth. Names,
  religions, personalities pass away, but the Law of Love remains. To
  become possessed of a knowledge of this Law, to enter into conscious
  harmony with it, is to become immortal, invincible, indestructible.
  
  It is because of the effort of the soul to realize this Law that men
  come again and again to live, to suffer, and to die; and when realized,
  suffering ceases, personality is dispersed, and the fleshly life and
  death are destroyed, for consciousness becomes one with the Eternal.
  
  The Law is absolutely impersonal, and its highest manifested expression
  is that of Service. When the purified heart has realized Truth it is
  then called upon to make the last, the greatest and holiest sacrifice,
  the sacrifice of the well-earned enjoyment of Truth. It is by virtue of
  this sacrifice that the divinely-emancipated soul comes to dwell among
  men, clothed with a body of flesh, content to dwell among the lowliest
  and least, and to be esteemed the servant of all mankind. That sublime
  humility which is manifested by the world's saviors is the seal of
  Godhead, and he who has annihilated the personality, and has become a
  living, visible manifestation of the impersonal, eternal, boundless
  Spirit of Love, is alone singled out as worthy to receive the unstinted
  worship of posterity. He only who succeeds in humbling himself with that
  divine humility which is not only the extinction of self, but is also
  the pouring out upon all the spirit of unselfish love, is exalted above
  measure, and given spiritual dominion in the hearts of mankind.
  
  All the great spiritual teachers have denied themselves personal
  luxuries, comforts, and rewards, have abjured temporal power, and have
  lived and taught the limitless and impersonal Truth. Compare their lives
  and teachings, and you will find the same simplicity, the same
  self-sacrifice, the same humility, love, and peace both lived and
  preached by them. They taught the same eternal Principles, the
  realization of which destroys all evil. Those who have been hailed and
  worshiped as the saviors of mankind are manifestations of the Great
  impersonal Law, and being such, were free from passion and prejudice,
  and having no opinions, and no special letter of doctrine to preach and
  defend, they never sought to convert and to proselytize. Living in the
  highest Goodness, the supreme Perfection, their sole object was to
  uplift mankind by manifesting that Goodness in thought, word, and deed.
  They stand between man the personal and God the impersonal, and serve as
  exemplary types for the salvation of self-enslaved mankind.
  
  Men who are immersed in self, and who cannot comprehend the Goodness
  that is absolutely impersonal, deny divinity to all saviors except their
  own, and thus introduce personal hatred and doctrinal controversy, and,
  while defending their own particular views with passion, look upon each
  other as being heathens or infidels, and so render null and void, as far
  as their lives are concerned, the unselfish beauty and holy grandeur of
  the lives and teachings of their own Masters. Truth cannot be limited;
  it can never be the special prerogative of any man, school, or nation,
  and when personality steps in, Truth is lost.
  
  The glory alike of the saint, the sage, and the savior is this,--that he
  has realized the most profound lowliness, the most sublime
  unselfishness; having given up all, even his own personality, all his
  works are holy and enduring, for they are freed from every taint of
  self. He gives, yet never thinks of receiving; he works without
  regretting the past or anticipating the future, and never looks for
  reward.
  
  When the farmer has tilled and dressed his land and put in the seed, he
  knows that he has done all that he can possibly do, and that now he must
  trust to the elements, and wait patiently for the course of time to
  bring about the harvest, and that no amount of expectancy on his part
  will affect the result. Even so, he who has realized Truth goes forth as
  a sower of the seeds of goodness, purity, love and peace, without
  expectancy, and never looking for results, knowing that there is the
  Great Over-ruling Law which brings about its own harvest in due time,
  and which is alike the source of preservation and destruction.
  
  Men, not understanding the divine simplicity of a profoundly unselfish
  heart, look upon their particular savior as the manifestation of a
  special miracle, as being something entirely apart and distinct from the
  nature of things, and as being, in his ethical excellence, eternally
  unapproachable by the whole of mankind. This attitude of unbelief (for
  such it is) in the divine perfectibility of man, paralyzes effort, and
  binds the souls of men as with strong ropes to sin and suffering. Jesus
  ``grew in wisdom'' and was ``perfected by suffering.'' What Jesus was,
  he became such; what Buddha was, he became such; and every holy man
  became such by unremitting perseverance in self-sacrifice. Once
  recognize this, once realize that by watchful effort and hopeful
  perseverance you can rise above your lower nature, and great and
  glorious will be the vistas of attainment that will open out before you.
  Buddha vowed that he would not relax his efforts until he arrived at the
  state of perfection, and he accomplished his purpose.
  
  What the saints, sages, and saviors have accomplished, you likewise may
  accomplish if you will only tread the way which they trod and pointed
  out, the way of self-sacrifice, of self-denying service.
  
  Truth is very simple. It says, ``Give up self,'' ``Come unto Me'' (away
  from all that defiles) ``and I will give you rest.'' All the mountains
  of commentary that have been piled upon it cannot hide it from the heart
  that is earnestly seeking for Righteousness. It does not require
  learning; it can be known in spite of learning. Disguised under many
  forms by erring self-seeking man, the beautiful simplicity and clear
  transparency of Truth remains unaltered and undimmed, and the unselfish
  heart enters into and partakes of its shining radiance. Not by weaving
  complex theories, not by building up speculative philosophies is Truth
  realized; but by weaving the web of inward purity, by building up the
  Temple of a stainless life is Truth realized.
  
  He who enters upon this holy way begins by restraining his passions.
  This is virtue, and is the beginning of saintship, and saintship is the
  beginning of holiness. The entirely worldly man gratifies all his
  desires, and practices no more restraint than the law of the land in
  which he lives demands; the virtuous man restrains his passions; the
  saint attacks the enemy of Truth in its stronghold within his own heart,
  and restrains all selfish and impure thoughts; while the holy man is he
  who is free from passion and all impure thought, and to whom goodness
  and purity have become as natural as scent and color are to the flower.
  The holy man is divinely wise; he alone knows Truth in its fullness, and
  has entered into abiding rest and peace. For him evil has ceased; it has
  disappeared in the universal light of the All-Good. Holiness is the
  badge of wisdom. Said Krishna to the Prince Arjuna--
  
  \begin{quote}
  ``Humbleness, truthfulness, and harmlessness, Patience and honor,
  reverence for the wise, Purity, constancy, control of self, Contempt of
  sense-delights, self-sacrifice, Perception of the certitude of ill In
  birth, death, age, disease, suffering and sin; An ever tranquil heart in
  fortunes good And fortunes evil, \ldots{} \ldots{} Endeavors resolute To
  reach perception of the utmost soul, And grace to understand what gain
  it were So to attain--this is true wisdom, Prince! And what is otherwise
  is ignorance!''
  \end{quote}
  
  Whoever fights ceaselessly against his own selfishness, and strives to
  supplant it with all-embracing love, is a saint, whether he live in a
  cottage or in the midst of riches and influence; or whether he preaches
  or remains obscure.
  
  To the worldling, who is beginning to aspire towards higher things, the
  saint, such as a sweet St.~Francis of Assisi, or a conquering
  St.~Anthony, is a glorious and inspiring spectacle; to the saint, an
  equally enrapturing sight is that of the sage, sitting serene and holy,
  the conqueror of sin and sorrow, no more tormented by regret and
  remorse, and whom even temptation can never reach; and yet even the sage
  is drawn on by a still more glorious vision, that of the savior actively
  manifesting his knowledge in selfless works, and rendering his divinity
  more potent for good by sinking himself in the throbbing, sorrowing,
  aspiring heart of mankind.
  
  And this only is true service--to forget oneself in love towards all, to
  lose oneself in working for the whole. O thou vain and foolish man, who
  thinkest that thy many works can save thee; who, chained to all error,
  talkest loudly of thyself, thy work, and thy many sacrifices, and
  magnifiest thine own importance; know this, that though thy fame fill
  the whole earth, all thy work shall come to dust, and thou thyself be
  reckoned lower than the least in the Kingdom of Truth!
  
  Only the work that is impersonal can live; the works of self are both
  powerless and perishable. Where duties, howsoever humble, are done
  without self-interest, and with joyful sacrifice, there is true service
  and enduring work. Where deeds, however brilliant and apparently
  successful, are done from love of self, there is ignorance of the Law of
  Service, and the work perishes.
  
  It is given to the world to learn one great and divine lesson, the
  lesson of absolute unselfishness. The saints, sages, and saviors of all
  time are they who have submitted themselves to this task, and have
  learned and lived it. All the Scriptures of the world are framed to
  teach this one lesson; all the great teachers reiterate it. It is too
  simple for the world which, scorning it, stumbles along in the complex
  ways of selfishness.
  
  A pure heart is the end of all religion and the beginning of divinity.
  To search for this Righteousness is to walk the Way of Truth and Peace,
  and he who enters this Way will soon perceive that Immortality which is
  independent of birth and death, and will realize that in the Divine
  economy of the universe the humblest effort is not lost.
  
  The divinity of a Krishna, a Gautama, or a Jesus is the crowning glory
  of self-abnegation, the end of the soul's pilgrimage in matter and
  mortality, and the world will not have finished its long journey until
  every soul has become as these, and has entered into the blissful
  realization of its own divinity.
  
  \begin{quote}
  Great glory crowns the heights of hope by arduous struggle won; Bright
  honor rounds the hoary head that mighty works hath done; Fair riches
  come to him who strives in ways of golden gain. And fame enshrines his
  name who works with genius-glowing brain; But greater glory waits for
  him who, in the bloodless strife 'Gainst self and wrong, adopts, in
  love, the sacrificial life; And brighter honor rounds the brow of him
  who, 'mid the scorns Of blind idolaters of self, accepts the crown of
  thorns; And fairer purer riches come to him who greatly strives To walk
  in ways of love and truth to sweeten human lives; And he who serveth
  well mankind exchanges fleeting fame For Light eternal, Joy and Peace,
  and robes of heavenly flame.
  \end{quote}
  
  \section{THE REALIZATION OF PERFECT
  PEACE}\label{the-realization-of-perfect-peace}
  
  In the external universe there is ceaseless turmoil, change, and unrest;
  at the heart of all things there is undisturbed repose; in this deep
  silence dwelleth the Eternal.
  
  Man partakes of this duality, and both the surface change and
  disquietude, and the deep-seated eternal abode of Peace, are contained
  within him.
  
  As there are silent depths in the ocean which the fiercest storm cannot
  reach, so there are silent, holy depths in the heart of man which the
  storms of sin and sorrow can never disturb. To reach this silence and to
  live consciously in it is peace.
  
  Discord is rife in the outward world, but unbroken harmony holds sway at
  the heart of the universe. The human soul, torn by discordant passion
  and grief, reaches blindly toward the harmony of the sinless state, and
  to reach this state and to live consciously in it is peace.
  
  Hatred severs human lives, fosters persecution, and hurls nations into
  ruthless war, yet men, though they do not understand why, retain some
  measure of faith in the overshadowing of a Perfect Love; and to reach
  this Love and to live consciously in it is peace.
  
  And this inward peace, this silence, this harmony, this Love, is the
  Kingdom of Heaven, which is so difficult to reach because few are
  willing to give up themselves and to become as little children.
  
  \begin{quote}
  ``Heaven's gate is very narrow and minute, It cannot be perceived by
  foolish men Blinded by vain illusions of the world; E'en the
  clear-sighted who discern the way, And seek to enter, find the portal
  barred, And hard to be unlocked. Its massive bolts Are pride and
  passion, avarice and lust.''
  \end{quote}
  
  Men cry peace! peace! where there is no peace, but on the contrary,
  discord, disquietude and strife. Apart from that Wisdom which is
  inseparable from self-renunciation, there can be no real and abiding
  peace.
  
  The peace which results from social comfort, passing gratification, or
  worldly victory is transitory in its nature, and is burnt up in the heat
  of fiery trial. Only the Peace of Heaven endures through all trial, and
  only the selfless heart can know the Peace of Heaven.
  
  Holiness alone is undying peace. Self-control leads to it, and the
  ever-increasing Light of Wisdom guides the pilgrim on his way. It is
  partaken of in a measure as soon as the path of virtue is entered upon,
  but it is only realized in its fullness when self disappears in the
  consummation of a stainless life.
  
  \begin{quote}
  ``This is peace, To conquer love of self and lust of life, To tear
  deep-rooted passion from the heart To still the inward strife.''
  \end{quote}
  
  If, O reader! you would realize the Light that never fades, the Joy that
  never ends, and the tranquillity that cannot be disturbed; if you would
  leave behind for ever your sins, your sorrows, your anxieties and
  perplexities; if, I say, you would partake of this salvation, this
  supremely glorious Life, then conquer yourself. Bring every thought,
  every impulse, every desire into perfect obedience to the divine power
  resident within you. There is no other way to peace but this, and if you
  refuse to walk it, your much praying and your strict adherence to ritual
  will be fruitless and unavailing, and neither gods nor angels can help
  you. Only to him that overcometh is given the white stone of the
  regenerate life, on which is written the New and Ineffable Name.
  
  Come away, for awhile, from external things, from the pleasures of the
  senses, from the arguments of the intellect, from the noise and the
  excitements of the world, and withdraw yourself into the inmost chamber
  of your heart, and there, free from the sacrilegious intrusion of all
  selfish desires, you will find a deep silence, a holy calm, a blissful
  repose, and if you will rest awhile in that holy place, and will
  meditate there, the faultless eye of Truth will open within you, and you
  will see things as they really are. This holy place within you is your
  real and eternal self; it is the divine within you; and only when you
  identify yourself with it can you be said to be ``clothed and in your
  right mind.'' It is the abode of peace, the temple of wisdom, the
  dwelling-place of immortality. Apart from this inward resting-place,
  this Mount of Vision, there can be no true peace, no knowledge of the
  Divine, and if you can remain there for one minute, one hour, or one
  day, it is possible for you to remain there always. All your sins and
  sorrows, your fears and anxieties are your own, and you can cling to
  them or you can give them up. Of your own accord you cling to your
  unrest; of your own accord you can come to abiding peace. No one else
  can give up sin for you; you must give it up yourself. The greatest
  teacher can do no more than walk the way of Truth for himself, and point
  it out to you; you yourself must walk it for yourself. You can obtain
  freedom and peace alone by your own efforts, by yielding up that which
  binds the soul, and which is destructive of peace.
  
  The angels of divine peace and joy are always at hand, and if you do not
  see them, and hear them, and dwell with them, it is because you shut
  yourself out from them, and prefer the company of the spirits of evil
  within you. You are what you will to be, what you wish to be, what you
  prefer to be. You can commence to purify yourself, and by so doing can
  arrive at peace, or you can refuse to purify yourself, and so remain
  with suffering.
  
  Step aside, then; come out of the fret and the fever of life; away from
  the scorching heat of self, and enter the inward resting-place where the
  cooling airs of peace will calm, renew, and restore you.
  
  Come out of the storms of sin and anguish. Why be troubled and
  tempest-tossed when the haven of Peace of God is yours!
  
  Give up all self-seeking; give up self, and lo! the Peace of God is
  yours!
  
  Subdue the animal within you; conquer every selfish uprising, every
  discordant voice; transmute the base metals of your selfish nature into
  the unalloyed gold of Love, and you shall realize the Life of Perfect
  Peace. Thus subduing, thus conquering, thus transmuting, you will, O
  reader! while living in the flesh, cross the dark waters of mortality,
  and will reach that Shore upon which the storms of sorrow never beat,
  and where sin and suffering and dark uncertainty cannot come. Standing
  upon that Shore, holy, compassionate, awakened, and self-possessed and
  glad with unending gladness, you will realize that
  
  \begin{quote}
  ``Never the Spirit was born, the Spirit will cease to be never; Never
  was time it was not, end and beginning are dreams; Birthless and
  deathless and changeless remaineth the Spirit for ever; Death hath not
  touched it at all, dead though the house of it seems.''
  \end{quote}
  
  You will then know the meaning of Sin, of Sorrow, of Suffering, and that
  the end thereof is Wisdom; will know the cause and the issue of
  existence.
  
  And with this realization you will enter into rest, for this is the
  bliss of immortality, this the unchangeable gladness, this the
  untrammeled knowledge, undefiled Wisdom, and undying Love; this, and
  this only, is the realization of Perfect Peace.
  
  \begin{quote}
  O thou who wouldst teach men of Truth! Hast thou passed through the
  desert of doubt? Art thou purged by the fires of sorrow? hath ruth The
  fiends of opinion cast out Of thy human heart? Is thy soul so fair That
  no false thought can ever harbor there?
  \end{quote}
  
  \begin{quote}
  O thou who wouldst teach men of Love! Hast thou passed through the place
  of despair? Hast thou wept through the dark night of grief? does it move
  (Now freed from its sorrow and care) Thy human heart to pitying
  gentleness, Looking on wrong, and hate, and ceaseless stress?
  \end{quote}
  
  \begin{quote}
  O thou who wouldst teach men of Peace! Hast thou crossed the wide ocean
  of strife? Hast thou found on the Shores of the Silence, Release from
  all the wild unrest of life? From thy human heart hath all striving
  gone, Leaving but Truth, and Love, and Peace alone?
  \end{quote}
  
  \section*{Postlude}\label{postlude}
  \addcontentsline{toc}{section}{Postlude}
  
  End of the Project Gutenberg EBook of The Way of Peace, by James Allen
  
  *** END OF THIS PROJECT GUTENBERG EBOOK THE WAY OF PEACE ***
  
  ***** This file should be named 10740-8.txt or 10740-8.zip ***** This
  and all associated files of various formats will be found in:
  http://www.gutenberg.net/1/0/7/4/10740/
  
  Produced by Kevin Handy and PG Distributed Proofreaders
  
  Updated editions will replace the previous one--the old editions will be
  renamed.
  
  Creating the works from public domain print editions means that no one
  owns a United States copyright in these works, so the Foundation (and
  you!) can copy and distribute it in the United States without permission
  and without paying copyright royalties. Special rules, set forth in the
  General Terms of Use part of this license, apply to copying and
  distributing Project Gutenberg-tm electronic works to protect the
  PROJECT GUTENBERG-tm concept and trademark. Project Gutenberg is a
  registered trademark, and may not be used if you charge for the eBooks,
  unless you receive specific permission. If you do not charge anything
  for copies of this eBook, complying with the rules is very easy. You may
  use this eBook for nearly any purpose such as creation of derivative
  works, reports, performances and research. They may be modified and
  printed and given away--you may do practically ANYTHING with public
  domain eBooks. Redistribution is subject to the trademark license,
  especially commercial redistribution.
  
  *** START: FULL LICENSE ***
  
  THE FULL PROJECT GUTENBERG LICENSE PLEASE READ THIS BEFORE YOU
  DISTRIBUTE OR USE THIS WORK
  
  To protect the Project Gutenberg-tm mission of promoting the free
  distribution of electronic works, by using or distributing this work (or
  any other work associated in any way with the phrase ``Project
  Gutenberg''), you agree to comply with all the terms of the Full Project
  Gutenberg-tm License (available with this file or online at
  http://gutenberg.net/license).
  
  Section 1. General Terms of Use and Redistributing Project Gutenberg-tm
  electronic works
  
  1.A. By reading or using any part of this Project Gutenberg-tm
  electronic work, you indicate that you have read, understand, agree to
  and accept all the terms of this license and intellectual property
  (trademark/copyright) agreement. If you do not agree to abide by all the
  terms of this agreement, you must cease using and return or destroy all
  copies of Project Gutenberg-tm electronic works in your possession. If
  you paid a fee for obtaining a copy of or access to a Project
  Gutenberg-tm electronic work and you do not agree to be bound by the
  terms of this agreement, you may obtain a refund from the person or
  entity to whom you paid the fee as set forth in paragraph 1.E.8.
  
  1.B. ``Project Gutenberg'' is a registered trademark. It may only be
  used on or associated in any way with an electronic work by people who
  agree to be bound by the terms of this agreement. There are a few things
  that you can do with most Project Gutenberg-tm electronic works even
  without complying with the full terms of this agreement. See paragraph
  1.C below. There are a lot of things you can do with Project
  Gutenberg-tm electronic works if you follow the terms of this agreement
  and help preserve free future access to Project Gutenberg-tm electronic
  works. See paragraph 1.E below.
  
  1.C. The Project Gutenberg Literary Archive Foundation (``the
  Foundation'' or PGLAF), owns a compilation copyright in the collection
  of Project Gutenberg-tm electronic works. Nearly all the individual
  works in the collection are in the public domain in the United States.
  If an individual work is in the public domain in the United States and
  you are located in the United States, we do not claim a right to prevent
  you from copying, distributing, performing, displaying or creating
  derivative works based on the work as long as all references to Project
  Gutenberg are removed. Of course, we hope that you will support the
  Project Gutenberg-tm mission of promoting free access to electronic
  works by freely sharing Project Gutenberg-tm works in compliance with
  the terms of this agreement for keeping the Project Gutenberg-tm name
  associated with the work. You can easily comply with the terms of this
  agreement by keeping this work in the same format with its attached full
  Project Gutenberg-tm License when you share it without charge with
  others.
  
  1.D. The copyright laws of the place where you are located also govern
  what you can do with this work. Copyright laws in most countries are in
  a constant state of change. If you are outside the United States, check
  the laws of your country in addition to the terms of this agreement
  before downloading, copying, displaying, performing, distributing or
  creating derivative works based on this work or any other Project
  Gutenberg-tm work. The Foundation makes no representations concerning
  the copyright status of any work in any country outside the United
  States.
  
  1.E. Unless you have removed all references to Project Gutenberg:
  
  1.E.1. The following sentence, with active links to, or other immediate
  access to, the full Project Gutenberg-tm License must appear prominently
  whenever any copy of a Project Gutenberg-tm work (any work on which the
  phrase ``Project Gutenberg'' appears, or with which the phrase ``Project
  Gutenberg'' is associated) is accessed, displayed, performed, viewed,
  copied or distributed:
  
  This eBook is for the use of anyone anywhere at no cost and with almost
  no restrictions whatsoever. You may copy it, give it away or re-use it
  under the terms of the Project Gutenberg License included with this
  eBook or online at www.gutenberg.net
  
  1.E.2. If an individual Project Gutenberg-tm electronic work is derived
  from the public domain (does not contain a notice indicating that it is
  posted with permission of the copyright holder), the work can be copied
  and distributed to anyone in the United States without paying any fees
  or charges. If you are redistributing or providing access to a work with
  the phrase ``Project Gutenberg'' associated with or appearing on the
  work, you must comply either with the requirements of paragraphs 1.E.1
  through 1.E.7 or obtain permission for the use of the work and the
  Project Gutenberg-tm trademark as set forth in paragraphs 1.E.8 or
  1.E.9.
  
  1.E.3. If an individual Project Gutenberg-tm electronic work is posted
  with the permission of the copyright holder, your use and distribution
  must comply with both paragraphs 1.E.1 through 1.E.7 and any additional
  terms imposed by the copyright holder. Additional terms will be linked
  to the Project Gutenberg-tm License for all works posted with the
  permission of the copyright holder found at the beginning of this work.
  
  1.E.4. Do not unlink or detach or remove the full Project Gutenberg-tm
  License terms from this work, or any files containing a part of this
  work or any other work associated with Project Gutenberg-tm.
  
  1.E.5. Do not copy, display, perform, distribute or redistribute this
  electronic work, or any part of this electronic work, without
  prominently displaying the sentence set forth in paragraph 1.E.1 with
  active links or immediate access to the full terms of the Project
  Gutenberg-tm License.
  
  1.E.6. You may convert to and distribute this work in any binary,
  compressed, marked up, nonproprietary or proprietary form, including any
  word processing or hypertext form. However, if you provide access to or
  distribute copies of a Project Gutenberg-tm work in a format other than
  ``Plain Vanilla ASCII'' or other format used in the official version
  posted on the official Project Gutenberg-tm web site
  (www.gutenberg.net), you must, at no additional cost, fee or expense to
  the user, provide a copy, a means of exporting a copy, or a means of
  obtaining a copy upon request, of the work in its original ``Plain
  Vanilla ASCII'' or other form. Any alternate format must include the
  full Project Gutenberg-tm License as specified in paragraph 1.E.1.
  
  1.E.7. Do not charge a fee for access to, viewing, displaying,
  performing, copying or distributing any Project Gutenberg-tm works
  unless you comply with paragraph 1.E.8 or 1.E.9.
  
  1.E.8. You may charge a reasonable fee for copies of or providing access
  to or distributing Project Gutenberg-tm electronic works provided that
  
  \begin{itemize}
  \item
    You pay a royalty fee of 20\% of the gross profits you derive from the
    use of Project Gutenberg-tm works calculated using the method you
    already use to calculate your applicable taxes. The fee is owed to the
    owner of the Project Gutenberg-tm trademark, but he has agreed to
    donate royalties under this paragraph to the Project Gutenberg
    Literary Archive Foundation. Royalty payments must be paid within 60
    days following each date on which you prepare (or are legally required
    to prepare) your periodic tax returns. Royalty payments should be
    clearly marked as such and sent to the Project Gutenberg Literary
    Archive Foundation at the address specified in Section 4,
    ``Information about donations to the Project Gutenberg Literary
    Archive Foundation.''
  \item
    You provide a full refund of any money paid by a user who notifies you
    in writing (or by e-mail) within 30 days of receipt that s/he does not
    agree to the terms of the full Project Gutenberg-tm License. You must
    require such a user to return or destroy all copies of the works
    possessed in a physical medium and discontinue all use of and all
    access to other copies of Project Gutenberg-tm works.
  \item
    You provide, in accordance with paragraph 1.F.3, a full refund of any
    money paid for a work or a replacement copy, if a defect in the
    electronic work is discovered and reported to you within 90 days of
    receipt of the work.
  \item
    You comply with all other terms of this agreement for free
    distribution of Project Gutenberg-tm works.
  \end{itemize}
  
  1.E.9. If you wish to charge a fee or distribute a Project Gutenberg-tm
  electronic work or group of works on different terms than are set forth
  in this agreement, you must obtain permission in writing from both the
  Project Gutenberg Literary Archive Foundation and Michael Hart, the
  owner of the Project Gutenberg-tm trademark. Contact the Foundation as
  set forth in Section 3 below.
  
  1.F.
  
  1.F.1. Project Gutenberg volunteers and employees expend considerable
  effort to identify, do copyright research on, transcribe and proofread
  public domain works in creating the Project Gutenberg-tm collection.
  Despite these efforts, Project Gutenberg-tm electronic works, and the
  medium on which they may be stored, may contain ``Defects,'' such as,
  but not limited to, incomplete, inaccurate or corrupt data,
  transcription errors, a copyright or other intellectual property
  infringement, a defective or damaged disk or other medium, a computer
  virus, or computer codes that damage or cannot be read by your
  equipment.
  
  1.F.2. LIMITED WARRANTY, DISCLAIMER OF DAMAGES - Except for the ``Right
  of Replacement or Refund'' described in paragraph 1.F.3, the Project
  Gutenberg Literary Archive Foundation, the owner of the Project
  Gutenberg-tm trademark, and any other party distributing a Project
  Gutenberg-tm electronic work under this agreement, disclaim all
  liability to you for damages, costs and expenses, including legal fees.
  YOU AGREE THAT YOU HAVE NO REMEDIES FOR NEGLIGENCE, STRICT LIABILITY,
  BREACH OF WARRANTY OR BREACH OF CONTRACT EXCEPT THOSE PROVIDED IN
  PARAGRAPH F3. YOU AGREE THAT THE FOUNDATION, THE TRADEMARK OWNER, AND
  ANY DISTRIBUTOR UNDER THIS AGREEMENT WILL NOT BE LIABLE TO YOU FOR
  ACTUAL, DIRECT, INDIRECT, CONSEQUENTIAL, PUNITIVE OR INCIDENTAL DAMAGES
  EVEN IF YOU GIVE NOTICE OF THE POSSIBILITY OF SUCH DAMAGE.
  
  1.F.3. LIMITED RIGHT OF REPLACEMENT OR REFUND - If you discover a defect
  in this electronic work within 90 days of receiving it, you can receive
  a refund of the money (if any) you paid for it by sending a written
  explanation to the person you received the work from. If you received
  the work on a physical medium, you must return the medium with your
  written explanation. The person or entity that provided you with the
  defective work may elect to provide a replacement copy in lieu of a
  refund. If you received the work electronically, the person or entity
  providing it to you may choose to give you a second opportunity to
  receive the work electronically in lieu of a refund. If the second copy
  is also defective, you may demand a refund in writing without further
  opportunities to fix the problem.
  
  1.F.4. Except for the limited right of replacement or refund set forth
  in paragraph 1.F.3, this work is provided to you `AS-IS' WITH NO OTHER
  WARRANTIES OF ANY KIND, EXPRESS OR IMPLIED, INCLUDING BUT NOT LIMITED TO
  WARRANTIES OF MERCHANTIBILITY OR FITNESS FOR ANY PURPOSE.
  
  1.F.5. Some states do not allow disclaimers of certain implied
  warranties or the exclusion or limitation of certain types of damages.
  If any disclaimer or limitation set forth in this agreement violates the
  law of the state applicable to this agreement, the agreement shall be
  interpreted to make the maximum disclaimer or limitation permitted by
  the applicable state law. The invalidity or unenforceability of any
  provision of this agreement shall not void the remaining provisions.
  
  1.F.6. INDEMNITY - You agree to indemnify and hold the Foundation, the
  trademark owner, any agent or employee of the Foundation, anyone
  providing copies of Project Gutenberg-tm electronic works in accordance
  with this agreement, and any volunteers associated with the production,
  promotion and distribution of Project Gutenberg-tm electronic works,
  harmless from all liability, costs and expenses, including legal fees,
  that arise directly or indirectly from any of the following which you do
  or cause to occur: (a) distribution of this or any Project Gutenberg-tm
  work, (b) alteration, modification, or additions or deletions to any
  Project Gutenberg-tm work, and (c) any Defect you cause.
  
  Section 2. Information about the Mission of Project Gutenberg-tm
  
  Project Gutenberg-tm is synonymous with the free distribution of
  electronic works in formats readable by the widest variety of computers
  including obsolete, old, middle-aged and new computers. It exists
  because of the efforts of hundreds of volunteers and donations from
  people in all walks of life.
  
  Volunteers and financial support to provide volunteers with the
  assistance they need, is critical to reaching Project Gutenberg-tm's
  goals and ensuring that the Project Gutenberg-tm collection will remain
  freely available for generations to come. In 2001, the Project Gutenberg
  Literary Archive Foundation was created to provide a secure and
  permanent future for Project Gutenberg-tm and future generations. To
  learn more about the Project Gutenberg Literary Archive Foundation and
  how your efforts and donations can help, see Sections 3 and 4 and the
  Foundation web page at http://www.pglaf.org.
  
  Section 3. Information about the Project Gutenberg Literary Archive
  Foundation
  
  The Project Gutenberg Literary Archive Foundation is a non profit
  501(c)(3) educational corporation organized under the laws of the state
  of Mississippi and granted tax exempt status by the Internal Revenue
  Service. The Foundation's EIN or federal tax identification number is
  64-6221541. Its 501(c)(3) letter is posted at
  http://pglaf.org/fundraising. Contributions to the Project Gutenberg
  Literary Archive Foundation are tax deductible to the full extent
  permitted by U.S. federal laws and your state's laws.
  
  The Foundation's principal office is located at 4557 Melan Dr.~S.
  Fairbanks, AK, 99712., but its volunteers and employees are scattered
  throughout numerous locations. Its business office is located at 809
  North 1500 West, Salt Lake City, UT 84116, (801) 596-1887, email
  business@pglaf.org. Email contact links and up to date contact
  information can be found at the Foundation's web site and official page
  at http://pglaf.org
  
  For additional contact information: Dr.~Gregory B. Newby Chief Executive
  and Director gbnewby@pglaf.org
  
  Section 4. Information about Donations to the Project Gutenberg Literary
  Archive Foundation
  
  Project Gutenberg-tm depends upon and cannot survive without wide spread
  public support and donations to carry out its mission of increasing the
  number of public domain and licensed works that can be freely
  distributed in machine readable form accessible by the widest array of
  equipment including outdated equipment. Many small donations (\$1 to
  \$5,000) are particularly important to maintaining tax exempt status
  with the IRS.
  
  The Foundation is committed to complying with the laws regulating
  charities and charitable donations in all 50 states of the United
  States. Compliance requirements are not uniform and it takes a
  considerable effort, much paperwork and many fees to meet and keep up
  with these requirements. We do not solicit donations in locations where
  we have not received written confirmation of compliance. To SEND
  DONATIONS or determine the status of compliance for any particular state
  visit http://pglaf.org
  
  While we cannot and do not solicit contributions from states where we
  have not met the solicitation requirements, we know of no prohibition
  against accepting unsolicited donations from donors in such states who
  approach us with offers to donate.
  
  International donations are gratefully accepted, but we cannot make any
  statements concerning tax treatment of donations received from outside
  the United States. U.S. laws alone swamp our small staff.
  
  Please check the Project Gutenberg Web pages for current donation
  methods and addresses. Donations are accepted in a number of other ways
  including including checks, online payments and credit card donations.
  To donate, please visit: http://pglaf.org/donate
  
  Section 5. General Information About Project Gutenberg-tm electronic
  works.
  
  Professor Michael S. Hart is the originator of the Project Gutenberg-tm
  concept of a library of electronic works that could be freely shared
  with anyone. For thirty years, he produced and distributed Project
  Gutenberg-tm eBooks with only a loose network of volunteer support.
  
  Project Gutenberg-tm eBooks are often created from several printed
  editions, all of which are confirmed as Public Domain in the U.S. unless
  a copyright notice is included. Thus, we do not necessarily keep eBooks
  in compliance with any particular paper edition.
  
  Each eBook is in a subdirectory of the same number as the eBook's eBook
  number, often in several formats including plain vanilla ASCII,
  compressed (zipped), HTML and others.
  
  Corrected EDITIONS of our eBooks replace the old file and take over the
  old filename and etext number. The replaced older file is renamed.
  VERSIONS based on separate sources are treated as new eBooks receiving
  new filenames and etext numbers.
  
  Most people start at our Web site which has the main PG search facility:
  
  \begin{verbatim}
   http://www.gutenberg.net
  \end{verbatim}
  
  This Web site includes information about Project Gutenberg-tm, including
  how to make donations to the Project Gutenberg Literary Archive
  Foundation, how to help produce our new eBooks, and how to subscribe to
  our email newsletter to hear about new eBooks.
  
  EBooks posted prior to November 2003, with eBook numbers BELOW \#10000,
  are filed in directories based on their release date. If you want to
  download any of these eBooks directly, rather than using the regular
  search system you may utilize the following addresses and just download
  by the etext year.
  
  \begin{verbatim}
   http://www.gutenberg.net/etext06
  
  (Or /etext 05, 04, 03, 02, 01, 00, 99,
   98, 97, 96, 95, 94, 93, 92, 92, 91 or 90)
  \end{verbatim}
  
  EBooks posted since November 2003, with etext numbers OVER \#10000, are
  filed in a different way. The year of a release date is no longer part
  of the directory path. The path is based on the etext number (which is
  identical to the filename). The path to the file is made up of single
  digits corresponding to all but the last digit in the filename. For
  example an eBook of filename 10234 would be found at:
  
  \begin{verbatim}
   http://www.gutenberg.net/1/0/2/3/10234
  \end{verbatim}
  
  or filename 24689 would be found at:
  http://www.gutenberg.net/2/4/6/8/24689
  
  An alternative method of locating eBooks:
  http://www.gutenberg.net/GUTINDEX.ALL
  
\end{document}
